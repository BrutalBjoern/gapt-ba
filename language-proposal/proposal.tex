\documentclass[11pt,a4paper]{article}

\usepackage[a4paper]{geometry}
\geometry{hmargin=2cm,vmargin=2cm}
\usepackage{setspace}
\singlespacing
\usepackage[small,compact]{titlesec}

\title{Stomping logical layers in GAPT}
\author{Daniel Weller}

\newcommand{\omicron}{o}

\newcommand{\ftype}{\omicron}
\newcommand{\itype}{\iota}
\newcommand{\atype}{\omega}
\newcommand{\impl}{\supset}
\newcommand{\cM}{\mathcal{M}}
\newcommand{\nmodels}{\not\models}

\begin{document}
%
\maketitle
%
\section{The problem}
%
GAPT currently supports 3 logical languages built on top of LambdaCalculus:
HOL, FOL, and schemata. Associated with the logical languages are
%
\begin{enumerate}
  \item Certain sets of lambda-terms (HOL formulas, first-order terms, etc.), and
  \item certain distinguished symbols (e.g.~$\top, \bot$, the successor symbol $s$ in schemata).
\end{enumerate}
%
The sets of lambda-terms are distinguished by the type information in scala. This is good: it allows
to convey some important information on the scala type-level (e.g.~application of a first-order
substitution is a function from first-order terms to first-order terms). The design of dynamically
``slapping on'' traits allows us to use general algorithms (e.g.~substitution) in specific cases, and
still have type-safety (with a single cast after the run of the algorithm).

But our current structure has problems:
%
\begin{enumerate}
  \item The logical symbols $\top, \lor, \ldots$ exist in multiple versions for the different layers.
  \item Each layer takes care of creating its own objects, which sometimes causes trouble since it
    is not possible to determine which layer should be responsible (e.g.~when creating the $\top$ symbol,
    we do not know whether it is ``first-order'' $\top$ or ``higher-order'' $\top$ without knowing
    some additional context).
\end{enumerate}
%
\section{The solution}
%
Instead of treating layers as independent entities that can be added ``by a user'', we assume that the
layer structure is a rarely changing part of GAPT that we control centrally. We assume that
all layers know everything about all other layers. This leads to the following design changes.
%
\begin{enumerate}
  \item We remove the different factories. There will be a single factory creating all lambda terms.
  \item The single factory is responsible for assessing which traits will be ``slapped onto'' a lambda
    term at creation.
  \item All distinguished constant symbols are collected and documented. Each algorithm is responsible
    for assessing which constant symbols it should treat as distinguished, and which as arbitrary (e.g.~the
    CNF transformation should distinguish $\top,\bot,\lor, \ldots$, but should treat the successor function
    symbol as an arbitrary function symbol). The same goes for distinguished lambda-types.
\end{enumerate}
%
\section{Distinguished lambda types}
%
We distinguish the types $\itype$ (uninterpreted individual type), $\ftype$ (boolean type),
$\atype$ (type of natural numbers). Intuitively, $\itype$ is the type of terms of uninterpreted
first-order logic, $\ftype$ the type of formulas, and $\atype$ the type of natural numbers (as, e.g.~in
a multi-sorted first-order setting).
%
\section{Distinguished constants}
%
The following table shows the distinguished constant symbols and their semantics in GAPT.

\vskip 0.3cm

\begin{tabular}{|l|l|l|}
\hline
Name & Type & Semantics\\
\hline
$\top$ & $\ftype$ & $\cM\models\top$\\
$\bot$ & $\ftype$ & $\cM\nmodels\top$\\
$\land$ & $\ftype\to\ftype\to\ftype$ &$\cM\models A \land B \Leftrightarrow \cM\models A\mbox{ and } \cM\models B$\\
$\lor$ & $\ftype\to\ftype\to\ftype$ &$\cM\models A \lor B \Leftrightarrow \cM\models A\mbox{ or } \cM\models B$\\
$\impl$ & $\ftype\to\ftype\to\ftype$ &$\cM\models A \impl B \Leftrightarrow \cM\nmodels A\mbox{ or } \cM\models B$\\
$\neg$ & $\ftype\to\ftype$ &$\cM\models \neg A \Leftrightarrow \cM\nmodels A$\\
$\forall_\alpha$ & $(\alpha \to \ftype)\to\ftype$ &$\cM\models \forall_\alpha(\lambda x.F(x)) \Leftrightarrow \cM\models F(o)\mbox{ for all }o\in\cM$\\
$\exists_\alpha$ & $(\alpha \to \ftype)\to\ftype$ &$\cM\models \exists_\alpha(\lambda x.F(x)) \Leftrightarrow \cM\models F(o)\mbox{ for some }o\in\cM$\\
$\bigwedge$ & $(\atype \to \ftype)\to\atype\to\atype\to\ftype$ &$\cM\models \bigwedge(\lambda x.F(x))lu \Leftrightarrow \cM\models F(n)\mbox{ for all }l\leq n\leq u\in\cM$\\
$\bigvee$ & $(\atype \to \ftype)\to\atype\to\atype\to\ftype$&$\cM\models \bigwedge(\lambda x.F(x))lu \Leftrightarrow \cM\models F(n)\mbox{ for some }l\leq n\leq u\in\cM$\\
\hline
\end{tabular}

\vskip 0.3cm

We will write $\forall$ instead of $\forall_\alpha$ if the type is unimportant.
We introduce the abbreviation $\forall x.F$ to mean $\forall(\lambda x.F)$. Analogously for $\exists$.
%
\section{Sets distinguished by traits}
%
We want to distinguish the following sets of lambda-terms. By ``distinguish'', we mean that every set
has an associated Scala trait, and that every LambdaExpression object in gapt should have exactly the
traits corresponding to the sets it belongs to.
%
\begin{enumerate}
  \item {\em Formulas} are lambda-terms of type $\ftype$.
  \item {\em First-order lambda-terms} are defined inductively: $\top,\bot,\land,\lor,\impl,\neg$ 
    are first-order lambda-terms,
    constant symbols of type $\itype \times \cdots \times \itype \to \itype$ (i.e.~function symbols)
    and $\itype\times\cdots\times \itype \to \ftype$ (i.e.~predicate symbols)
    are first-order lambda-terms. If $t, s$ are first-order lambda terms, then $ts$ is a first-order lambda term.
    If $F$ is a first-order lambda term and a formula, and $x$ a variable of type $\itype$,
    then $\forall x.F, \exists x.F$ are first-order lambda-terms.
  \item {\em First-order terms} are first-order lambda-terms of type $\itype$.
  \item {\em First-order formulas} are first-order lambda-terms of type $\ftype$.
  \item {\em Propositional atoms} are constant symbols of type $\ftype$.
  \item {\em Propositional formulas} are defined inductively: If $F, G$ are propositional 
    formulas, then $F\circ G$ are propositional formulas for $\circ\in\{\land,\lor,\impl,\neg\}$.
  \item {\em Schematic first-order lambda-terms} are defined inductively:
    $\top,\bot,\land,\lor,\impl,\neg$ are schematic first-order
    lambda-terms, constant symbol of types $\tau_1\times \cdots \times \tau_n \to \tau_{n+1}$
    and $\tau_1\times \cdots \times \tau_n \to \ftype$, with $\tau_i\in\{\itype,\atype\}$, 
    are schematic first-order lambda-terms. If $t, s$ are schematic first-order lambda terms, then $ts$ is a schematic
    first-order lambda term. If $F$ is a schematic first-order lambda term and a formula, and $x$ a variable of type 
    $\itype$, then $\forall x.F, \exists x.F$ are schematic first-order lambda terms. If $t,s$ are terms of type
    $\atype$, $x$ a variable of type $\atype$, and $F$ a schematic first-order lambda term and a formula, then
    $\bigwedge(\lambda x.F)ts$ and $\bigvee(\lambda x.F)ts$ are schematic first-order lambda terms.
   \end{enumerate}
%
For sets of lambda-terms $S,T$, we can encode the relation $S\subseteq T$ in Scala by
having the trait for $S$ extend the trait for $T$.
The following subset relations are encoded in GAPT: 
Schematic first-order formulas $\subseteq$ formulas.
First-order formulas $\subseteq$ schematic first-order formulas.
Propositional formulas $\subseteq$ first-order formulas.
Propositional atoms $\subseteq$ propositional formulas.
\end{document}
