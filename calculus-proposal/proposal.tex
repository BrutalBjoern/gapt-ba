\documentclass[11pt,a4paper]{article}

\usepackage[a4paper]{geometry}
\geometry{hmargin=2cm,vmargin=2cm}
\usepackage{setspace}
\singlespacing
\usepackage[small,compact]{titlesec}
\usepackage{proof}
\usepackage{amsmath}
\usepackage{listings}

\title{Reforming logical calculi in GAPT}
\author{Sebastian Zivota}

% "define" Scala
\lstdefinelanguage{scala}{
  morekeywords={abstract,case,catch,class,def,%
    do,else,extends,false,final,finally,%
    for,if,implicit,import,match,mixin,%
    new,null,object,override,package,%
    private,protected,requires,return,sealed,%
    super,this,throw,trait,true,try,%
    type,val,var,while,with,yield},
  otherkeywords={=>,<-,<\%,<:,>:,\#,@},
  sensitive=true,
  morecomment=[l]{//},
  morecomment=[n]{/*}{*/},
  morestring=[b]",
  morestring=[b]',
  morestring=[b]"""
}

\usepackage{color}
\definecolor{dkgreen}{rgb}{0,0.6,0}
\definecolor{gray}{rgb}{0.5,0.5,0.5}
\definecolor{mauve}{rgb}{0.58,0,0.82}

\lstset{frame=tb,
  language=scala,
  aboveskip=3mm,
  belowskip=3mm,
  showstringspaces=false,
  columns=flexible,
  basicstyle={\small\ttfamily},
  numbers=none,
  numberstyle=\tiny\color{gray},
  keywordstyle=\color{blue},
  commentstyle=\color{dkgreen},
  stringstyle=\color{mauve},
  frame=single,
  breaklines=true,
  breakatwhitespace=true
  tabsize=3
}

\begin{document}
%
\maketitle
%
\section{General changes}
\begin{itemize}
 \item The abstract proof/tree infrastructure is replaced by a trait Tree that allows access to children and subtrees (perhaps via a dedicated position class?) and supports operations such as mapping. No checking acyclicity and such.
 \item Individual LK rules are case classes extending type LKProof.
 \item The FormulaOccurrence class is eliminated. All proofs operate on sequents of formulas.
 \item The auxiliar formulas of rules are referred to by SequentIndices.
 \item The main formulas of rules are always created at the outermost position, i.e. on the left of the antecedent or the right of the succedent (the +: and :+ operators for sequents already woverbatimrk in this manner).
 \item The main constructor of a rule accepts the minimal information required to construct the rule top-down (i.e. subproofs and indices of auxiliar formulas, possibly term positions, main formula in the case of weakening etc).
 \item There are various convenience constructors that allow the construction of a rule without reference to specific formula occurrences.
\end{itemize}

\section{Individual rules}
\begin{itemize}
\item The current Axiom class is replaced by the class InitialSequent, which captures any kind of sequent that ends a proof. Extending it are classes like LogicalAxiom, which models an axiom of the form $A \vdash A$ with A atomic, and EqualityAxiom, which models an axiom of the form $\vdash s = s$.
\item Binary equality rules are replaced by unary ones. Paramodulation is realized as an equality rule followed by a cut.
\item Unary propositional rules are made multiplicative to agree with the rest of the calculus.
\item Proper verification for quantifier rules.
\end{itemize}

\section{Occurrences}
Eliminating the FormulaOccurrence class leaves us with the problem of tracking the ancestry of a formula through a proof. Fortunately, due to the way rules are constructed (the main formula is on the outside, all other formulas are copied in order), finding the parents of a formula is essentially an arithmetic problem. Suppose we have a proof $p$ ending in a unary rule. $p$ has a method getOccurrenceConnector that returns an object of type OccurrenceConnector. This object models the connection between formula occurrences int the premise and conclusion of the rule. It provides two methods, parents(i: SequentIndex): Seq[SequentIndex] and child(i: SequentIndex): Option[SequentIndex] that return the list of parents of a formula occurrence in the conclusion and the (possibly non-existent) child of an occurrence in the premise, respectively. These methods are implemented ad-hoc based on the rule under consideration.

As an example, consider a $c_l$ rule:
\begin{align*}
 p = \vcenter{\infer[c_l(i,j)]
 {A, \Gamma_1, \Gamma_2, \Gamma_3 \vdash \Delta}
 {\Gamma_1, A_{[i]}, \Gamma_2, A_{[j]}, \Gamma_3 \vdash \Delta}}
\end{align*}
The code would look something like this:
\begin{lstlisting}
 case class ContractionLeftRule(premise: HOLSequent, i: Int, j: Int) extends LKProof {
 
  // Do the basic construction of the rule
  
  override def getOccurrenceConnector: OccurrenceConnector = new OccurrenceConnector {
    override def parents(idx: SequentIndex): Seq[SequentIndex] = {
      val (m, n) = endSequent.lengths
      idx match {
	case Ant(0) =>  Seq(Ant(i), Ant(j)) // main formula has aux formulas as ancestors
	case Ant(k) if k <= i => Seq(Ant(k-1)) // a formula in \Gamma_1
	case Ant(k) if k < j => Seq(Ant(k)) // a formula in \Gamma_2
	case Ant(k) if k < m => Seq(Ant(k+1)) // a formula in \Gamma_3
	case Ant(_)  => throw new Exception("Index too big")
	case Suc(k) if k < n => Suc(k) // formulas in the succedent are trivial
	case Suc(_) => throw new Exception("Index too big")
      }
    }
    
    override def child(idx: SequentIndex): Option[SequentIndex] = ...
  }
    
\end{lstlisting}

Analogously, a binary rule would have methods getLeftOccurrenceConnector and getRightOccurrenceConnector.

\section{Questions}
\begin{itemize}
 \item Should the auxiliar formulas of a rule be given as SequentIndices or plain Ints? The information whether the aux formulas are in the antecedent or in the succedent is already included in the type of the rule, at least most of the time.
\end{itemize}



\end{document}
