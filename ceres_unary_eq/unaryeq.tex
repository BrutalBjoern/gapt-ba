\documentclass[a4paper]{article}

\usepackage[usenames,dvipsnames,svgnames]{xcolor}
\usepackage{llk}
\usepackage{amsmath}
\usepackage{amsfonts}
\usepackage{amsthm}
\usepackage{mathabx}
\usepackage{fullpage}
\definecolor{linkcolor}{rgb}{0.1,0.0,0.35}
\definecolor{citecolor}{rgb}{0.1,0.0,0.35}
\definecolor{unchangedcolor}{rgb}{0.4,0.4,0.4}
\definecolor{changedcolor}{rgb}{1.0,0.2,0.0}
\usepackage[pdftex,
            colorlinks=true,
            linkcolor=linkcolor,
            filecolor=red,
            citecolor=citecolor,
            bookmarks, bookmarksnumbered=true]{hyperref}

\newenvironment{scprooftree}[1]%
  {\gdef\scalefactor{#1}\begin{center}\proofSkipAmount \leavevmode}%
  {\scalebox{\scalefactor}{\DisplayProof}\proofSkipAmount \end{center} 
}

\newcommand{\ccs}[0]{\ensuremath{\mathcal{C}\mathcal{S}}}
\newcommand{\proj}[0]{\ensuremath{\mathcal{P}}}
\newcommand{\set}[1]{\ensuremath{\{#1\}}}
\newcommand{\SEQUL}{\RL{=:ul}}
\newcommand{\SEQUR}{\RL{=:ur}}
\newcommand{\UEQR}[2]{\SEQUR \UI{#1}{#2}}
\newcommand{\UEQL}[2]{\SEQUL \UI{#1}{#2}}

\newcommand{\red}[1]{\textcolor{BrickRed}{#1}}
\newcommand{\blue}[1]{\textcolor{MidnightBlue}{#1}}

\setlength\parindent{0pt}

\begin{document}
\section{Rules}
We mark the ancestors of the occurence of the primary formula $F(t)$ red
 and we mark the ancestors of the occurence of the primary formula $s=t$ blue.

\subsection{Unary eq rule}
\begin{prooftree}
\AX{\Gamma, \blue{s=t}}{\red{F(s)}, \Delta}
\UEQR{\Gamma, \blue{s=t}}{\red{F(t)}, \Delta}
\end{prooftree}

\subsection{Binary eq rule}
\begin{prooftree}
\AX{\Gamma_1}{\red{s=t}, \Delta_1}
\AX{\Gamma_2}{\red{F(s)}, \Delta_2}
\EQR{\Gamma_1, \Gamma_2}{\red{F(t)}, \Delta_1,\Delta_2}
\end{prooftree}

\subsection{Binary rule by Unary rule}
\label{sec:binaryrule}
\begin{prooftree}
\AX{\Gamma_1}{s=t, \Delta_1}
\AX{\Gamma_2}{ \red{F(s)}, \Delta_2}
\WEAKL{\Gamma_2,\red{s=t}}{ \red{F(s)}, \Delta_2}
\UEQR{\Gamma_2,\red{s=t}}{ \red{F(t)}, \Delta_2}
\CUT{\Gamma_1,\Gamma_2}{\red{F(t)}, \Delta_1,\Delta_2}
\end{prooftree}

\subsection{Unary rule by Binary rule}
\label{sec:unaryrule}
\begin{prooftree}

\AX{\blue{s=t}}{\red{s=t}}
\AX{\Gamma,\blue{s=t}}{\red{F(s)},\Delta}
\EQR{\Gamma,\blue{s=t},\blue{s=t}}{\red{F(t)},\Delta}
\CONTRR{\Gamma,\blue{s=t}}{\red{F(t)},\Delta}
\end{prooftree}

%Alternative, if not restricted to tautology axioms:\\
%\begin{prooftree}
%
%\AX{}{\red{s=t}}
%\AX{\Gamma,\blue{s=t}}{\red{F(s)},\Delta}
%\EQR{\Gamma,\blue{s=t}}{\red{F(t)},\Delta}
%\end{prooftree}

\subsection{Inserting rule \ref{sec:unaryrule} into rule \ref{sec:binaryrule}}
\begin{prooftree}

\AX{\Gamma_1}{s=t, \Delta_1}
\AX{s=t}{\red{s=t}}
\AX{\Gamma_2}{ \red{F(s)}, \Delta_2}
\WEAKL{\Gamma_2,s=t}{ \red{F(s)}, \Delta_2}

\EQR{\Gamma_2,s=t,\red{s=t}}{\red{F(t)},\Delta_2}
\CONTRR{\Gamma_2,s=t}{\red{F(t)},\Delta_2}

\CUT{\Gamma_1,\Gamma_2}{\red{F(t)}, \Delta_1,\Delta_2}
\end{prooftree}

\subsection{Abbreviations}
\begin{tabular}{ll}
$\pi_1:$ & proof of \SEQUENT{\Gamma, s=t}{ \Delta}\\
$\pi_2:$ & proof of \SEQUENT{\Gamma}{F(t), \Delta}\\
$\pi_3:$ & proof of \SEQUENT{\Gamma,s=t}{F(s), \Delta}\\
$\pi_4:$ & proof of \SEQUENT{s=t}{s=t}\\
$S_i$ & struct of $\pi_i$\\
$C_i$ & \ccs{} of struct $S_i$ \\
$\proj_i$ & projection of $\pi_i$ \\
$P_i$ & end-sequent of $\proj_i$ \\
$S:$  & end-sequent \\
$Proj_n$ & set of projections after $n$ inferences\\
\end{tabular}


\section{Proposed \ccs{} and Projection semantics of $=:ur$}
We use the simulation of the unary rule by the binary rule to be able to use the
 extraction methods. Since $CERES_\omega$ has only axiom rules of the form
 $F \lkproves F$, we consider the first version of rule $\ref{sec:unaryrule}$.\\

Remark: In the axiom $\blue{s=t} \lkproves \red{s=t} $, the succedent formula is
 a cut-ancestor iff $\red{F(t)}$ is.
% If the antecedent $s=t$ is a cut-ancestor, the
% corresponding clause set $C_4$ is $\set{s=t \lkproves}$, otherwise $C_4$ is
% $\set{\lkproves}$.\\

We differentiate on the cut-ancestorship of $\red{F(t)}$ and $\blue{s=t}$ in the
 end-sequent of rule \ref{sec:unaryrule}.

\begin{enumerate}
\item if $F(t)$ is a cut-ancestor:
 Then the struct is $S_3 \oplus S_4$ and therefore the \ccs{} is $C_3 \cup C_4$.
  \begin{enumerate}
  \item If $\blue{s=t}$ is a cut-ancestor:
    Then $C_4=\set{}$ and the new \ccs{} is
    $C_3 \cup \set{} = C_3 $.
    The projections are unchanged, i.e. $Proj_{n+1} = Proj_n$.

  \item If $\blue{s=t}$ is an es-ancestor:
    Then $C_4=\set{ \lkproves \red{s=t}}$ and the new \ccs{} is
    $C_3 \cup \set{\lkproves \red{s=t}}$.
    Since
    \begin{prooftree}
      \AX{s=t}{s=t}
      \CONTINUEWITH{EQ_{s,t}}
    \end{prooftree}

    is an LK axiom, we can add it to the set
    of projections i.e. $Proj_{n+1} = Proj_n \cup EQ_{s,t}$
  \end{enumerate}

\item if $F(t)$ is an end-sequent ancestor:
  Then the struct is $S_3 \otimes S_4$ and therefore the \ccs{} is
  $C_3 \times C_4$.
  \begin{enumerate}
  \item If $\blue{s=t}$ is a cut-ancestor:
    \label{opt1}
    Since $s=t$ is a cut-ancestor, it is not necessarily part of the end-sequent
    (e.g. it might have been introduced by weakening or another equality
    inference) and we will denote its occurence as $F_{s,t}$\footnote{it may be
      not present at all if it comes from a weakening rule}. But the simulation
    of the rule (\ref{sec:unaryrule}) artificially introduces a second $s=t$
    which is later contracted. Instead, we can weaken first and skip the
    contraction.\\

    Then $C_4=\set{\blue{s=t} \lkproves}$ and the new \ccs{} is
    $C_3 \times \set{\blue{s=t} \lkproves} $. Then we apply the following
    inference:\\
    \begin{prooftree}
      \CONTINUEFROM{\proj_i}{\Lambda, F_{s,t} }{\Pi, F(s)}
      \WEAKL{\Lambda, \blue{s=t}, F_{s,t}}{\Pi, F(s)}
      \UEQR{\Lambda, \blue{s=t}, F_{s,t}}{\Pi, F(t)}
    \end{prooftree}
    to each $\proj_i \in Proj_n$.
  \item If $\blue{s=t}$ is an es-ancestor:
    Then $C_4=\set{ \lkproves}$ and the new \ccs{} is
    $C_3 \times \set{\lkproves} = C_3$.
    The equation rule is repeated, i.e. we infer
    \begin{prooftree}
      \CONTINUEFROM{\proj_i}{\Gamma, s=t}{F(s), \Delta}
      \UEQR{\Gamma, s=t}{F(t), \Delta}
      \CONTINUEWITH{\proj_i'}
    \end{prooftree}
    for $\proj_i \in Proj_n, \proj_i' \in Proj_{n+1}$.
  \end{enumerate}
\end{enumerate}

\section{Optimizations}
In case \ref{opt1}, the projection contains two cut-ancestor instances of $s=t$.
Since $\Gamma, s=t, s=t \lkproves \Delta$ is only refutable iff
$\Gamma, s=t \lkproves \Delta$ is, we can set $Proj_{n+1} = Proj_n$ and
 $C_4' = C_3$.


\end{document}
