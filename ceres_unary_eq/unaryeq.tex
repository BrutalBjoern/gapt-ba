\documentclass[a4paper]{article}

\usepackage{xcolor}
\usepackage{llk}
\usepackage{amsmath}
\usepackage{amsfonts}
\usepackage{amsthm}
\usepackage{mathabx}
\usepackage{fullpage}
\definecolor{linkcolor}{rgb}{0.1,0.0,0.35}
\definecolor{citecolor}{rgb}{0.1,0.0,0.35}
\definecolor{unchangedcolor}{rgb}{0.4,0.4,0.4}
\definecolor{changedcolor}{rgb}{1.0,0.2,0.0}
\usepackage[pdftex,
            colorlinks=true,
            linkcolor=linkcolor,
            filecolor=red,
            citecolor=citecolor,
            bookmarks, bookmarksnumbered=true]{hyperref}

\newenvironment{scprooftree}[1]%
  {\gdef\scalefactor{#1}\begin{center}\proofSkipAmount \leavevmode}%
  {\scalebox{\scalefactor}{\DisplayProof}\proofSkipAmount \end{center} 
}

\newcommand{\ccs}[0]{\ensuremath{\mathcal{C}\mathcal{S}}}
\newcommand{\set}[1]{\ensuremath{\{#1\}}}
\newcommand{\SEQUL}{\RL{=:ul}}
\newcommand{\SEQUR}{\RL{=:ur}}
\newcommand{\UEQR}[2]{\SEQUR \UI{#1}{#2}}
\newcommand{\UEQL}[2]{\SEQUL \UI{#1}{#2}}

\begin{document}
\section{Rules}


Binary eq rule:\\
\begin{prooftree}
\AX{\Gamma_1}{s=t, \Delta_1}
\AX{\Gamma_2}{F(s), \Delta_2}
\EQR{\Gamma_1, \Gamma_2}{F(t), \Delta_1,\Delta_2}
\end{prooftree}

Unary translation:\\
\begin{prooftree}
\AX{\Gamma_1}{s=t, \Delta_1}
\AX{\Gamma_2}{ F(s), \Delta_2}
\WEAKL{\Gamma_2,s=t}{ F(s), \Delta_2}
\UEQR{\Gamma_2,s=t}{ F(t), \Delta_2}
\CUT{\Gamma_1,\Gamma_2}{F(t), \Delta_1,\Delta_2}
\end{prooftree}

Binary back-translation:\\
\begin{prooftree}

\AX{s=t}{s=t}
\AX{\Gamma,s=t}{F(s),\Delta}
\EQR{\Gamma,s=t}{F(t),\Delta}
\end{prooftree}

Inserting back-translation into unary translation:
\begin{prooftree}

\AX{\Gamma_1}{s=t, \Delta_1}
\AX{s=t}{s=t}
\AX{\Gamma_2}{ F(s), \Delta_2}
\WEAKL{\Gamma_2,s=t}{ F(s), \Delta_2}

\EQR{\Gamma_2,s=t}{F(t),\Delta_2}

\CUT{\Gamma_1,\Gamma_2}{F(t), \Delta_1,\Delta_2}
\end{prooftree}


\begin{tabular}{ll}
$\pi_1:$ & proof of \SEQUENT{\Gamma, s=t}{ \Delta}\\
$\pi_2:$ & proof of \SEQUENT{\Gamma}{F(t), \Delta}\\
$\pi_3:$ & proof of \SEQUENT{\Gamma,s=t}{F(s), \Delta}\\
$\pi_4:$ & proof of \SEQUENT{s=t}{s=t}\\
$S_i$ & struct of $\pi_i$\\
$C_i$ & \ccs{} of struct $S_i$ \\
$P_i$ & projection of $\pi_i$ \\
\end{tabular}
\\[2mm]
\noindent
%Binary Struct:
%$S_1 \times S_2$ if $F(s)$ is not a cut-ancestor, $S_1 \cup S_2$ otherwise
%
%\noindent
%Unary Struct:
%$S_1 \cup S_2$


\section{Proposed \ccs{} semantics of $=:ur$}
Remark: In the axiom $s=t \lkproves s=t $, the succedent formula is never a cut-ancestor. If the antecedent $s=t$ is a cut-ancestor, the corresponding clause set $C_4$ is $\set{s=t \lkproves}$, otherwise $C_4$ is $\set{\lkproves}$.\\

\begin{itemize}
\item if $F(t)$ is a cut-ancestor: Then the stuct is $S_3 \oplus S_4$ and therefore the \ccs{} is $C_3 \cup C_4$.
  \begin{itemize}
  \item If $s=t$ is a cut-ancestor: Then $C_4={s=t \lkproves}$ and the new \ccs{} is $C_3 \cup \set{s=t \lkproves} $
  \item If $s=t$ is an es-ancestor: Then $C_4={ \lkproves}$ and the new \ccs{} is $C_3 \cup \set{\lkproves} $
  \end{itemize}
 
\item if $F(t)$ is an end-sequent ancestor: Then the stuct is $S_3 \otimes S_4$ and therefore the \ccs{} is $C_3 \times C_4$.
  \item If $s=t$ is a cut-ancestor: Then $C_4={s=t \lkproves}$ and the new \ccs{} is $C_3 \times \set{s=t \lkproves} $
  \item If $s=t$ is an es-ancestor: Then $C_4={ \lkproves}$ and the new \ccs{} is $C_3 \times \set{\lkproves} = C_3$
\end{itemize}



\end{document}
