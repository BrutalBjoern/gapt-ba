\documentclass[a4paper, 11pt]{report}
\usepackage[utf8]{inputenc}
%\usepackage{listings}
\usepackage{color}
\usepackage{amsmath}
\usepackage{url}
\usepackage{alltt}
\usepackage{multicol}
\usepackage{amssymb}
\usepackage{amsthm}
\usepackage{proof}

\usepackage{bussproofs}

\newcommand{\seq}{\vdash} % the sequent sign
\newcommand{\impl}{\supset} %logical connectives: implies, not, and, or
\renewcommand{\lnot}{\neg}
\renewcommand{\land}{\wedge}
\renewcommand{\lor}{\vee}

%Commands for constructing proof trees with bussproofs. See the chapter on the LK system for examples.
\newcommand{\UnaryInfCm}[1]{\UnaryInfC{$#1$}}
\newcommand{\BinaryInfCm}[1]{\BinaryInfC{$#1$}}
\newcommand{\RightLabelm}[1]{\RightLabel{$#1$}}
\newcommand{\AxiomCm}[1]{\AxiomC{$#1$}}

%Normal text in math mode ("math text")
\newcommand{\mt}[1]{\textnormal{#1}}

% Using a report without chapters
\renewcommand*\thesection{\arabic{section}}

%Theorem environments
\newtheorem{theorem}{Theorem}
\newtheorem{definition}{Definition}
\newtheorem{corollary}{Corollary}
\newtheorem{lemma}{Lemma}

%\lstset{
%  basicstyle=\footnotesize\ttfamily,
%  breaklines=true,
%  frame=lines
%}

\title{Cut Introduction\\ \vspace{0.5cm} Cut-Formulas with Multiple Universal Quantifiers}

\author{Janos Tapolczai}

\begin{document}
\maketitle

\section{Introduction}

In \cite[Ch. 5]{cutIntro2013}, algorithmic cut-introduction is described, albeit restricted to cut-formulas with
one universal quantifier. This document extends the mechanism described therein to cut-formulas with an
arbitrary number of universal quantifiers.

\section{Generalized $\Delta$-Vector}

One need only change the definition of the $\delta$-vector and extend it to deal with vectors of variables
instead of a single variable.\\

\noindent
First, we define the helper-function \texttt{nub}\footnote{\texttt{nub} is the common name for duplicate elimination in functional languages.} which eliminates duplicates:

$$
  \begin{array}{l}
    \texttt{nub}(f(u_1,\dots,u_m), (\overline{s_1},\dots,\overline{s_n})) = \texttt{elim} \uparrow \uparrow \infty =
    (f(u_1',\dots,u_m'), (\overline{s_1},\dots,\overline{s_p}))\\

    \\

    \quad 
      \begin{array}{l l}
        \mt{where } &
        \alpha_{F+1} \mt{ is the } \alpha \mt{ with the lowest index which occurs in } u_1,\dots,u_m \mt{ and}\\

        \\

        & \texttt{elim}(f(u_1,\dots,u_m), (\overline{s_1},\dots,\overline{s_n}) =\\
        & \mt{if } [\exists i,j: i < j]\ \overline{s_{i}} = \overline{s_{j}} \mt { then}\\
        & \quad \mt{remove } \overline{s_{j}},\\
        & \quad \mt{replace } \alpha_{j+F} \mt { with } \alpha_{i+F} \mt{ and }\\
        & \quad \mt{replace } \alpha_k \mt{ with } \alpha_{k-1}\ \forall k > j+F \mt{ in } u_1,\dots,u_m \mt{.}
        \\
      \end{array}
  \end{array}
$$

\noindent
We then define the generalized delta-vector $\Delta_G$:

$$
  \Delta_G(t_1,\dots,t_n) = \left\{
    \begin{array}{l}

    (t_1,()) \mt{ if } t_1 = t_2 = \dots = t_n \mt{ and } n > 0\\
    \\

    \texttt{nub}(f(u_1,\dots,u_m), (\overline{s_1},\dots,\overline{s_q})) %= (f(u_1',\dots,u_m'), (\overline{s_1},\dots,\overline{s_p}))
    \\
    \quad
    \begin{array}{l l}
      \mt{if all } & t_i = f(t_1^i,\dots,t_m^i) \mt{ and case 1 does not apply,}\\
      \mt{where } & (\overline{s_1},\dots,\overline{s_q}) = \bigsqcup\limits_{1 \leq j \leq m} \pi_2(\Delta_G(t_j^1,\dots,t_j^n)) \mt{ and}\\
                  & u_j = \pi_1(\Delta_G(t_j^1,\dots,t_j^n)) \mt{ for all } j \in \{1,\dots,m \}\\
    \end{array}
    
    \\
    \\
    
    (\alpha_{\mt{UNIQUE}},(t_1,\dots,t_n))\quad \mt{otherwise}\\
    \end{array}
  \right.
$$

\noindent
where $\sqcup$ is list concatenation, $\pi$ is the tuple projection function and $\alpha_{\mt{UNIQUE}}$ denotes an instance of $\alpha$ with a globally unique index that starts with 1 and is incremented by 1 with each instantiation left-to-right --- formally, if $\Delta_G(t_1,\dots,t_n) = (u,S)$, then the leftmost instance of $\alpha$ in $u$ is $\alpha_1$ and $\alpha_{j_1}$ is the nearest left sibling of $\alpha_{j_2}$ iff $j_1+1=j_2$.\\

Observe that, due to the incremental assignment of indices to generate $\alpha$-instances, $f(u_1,\dots,u_m)$ always contains a contiguous set $\{\alpha_k,\dots,\alpha_{k+q}\}$ (for a priori unknown $k$ and $q$).\\

Strictly speaking, neither applying \texttt{nub} would nor case 1 of $\Delta_G$ would be necessary, but they minimize the number of $\alpha$-instances: \texttt{nub} merges two $\alpha$-instances {\em if their respective $s$-vectors are identical}, whereas case 1 eliminates an $\alpha$-instance {\em if all terms within its $s$-vector} are identical.

\section{Behavior}

$\Delta_G$ is a generalization of $\Delta$ in a certain sense, although not a strict one. Both try to find the maximal common structure in $t_1,\dots,t_n$, but the ability to use an unbounded number of $\alpha$-instances instead of just one means that $\Delta_G$ will recursively descend until it encounters a termset in which at least two terms have different heads, whereas $\Delta$ will introduce its $\alpha$ as soon it is called on two different termsets where not all heads are equal, even if the heads {\em within} each termset are.

\section{Soundness}

\begin{theorem}
\textbf{Soundness of $\Delta_G$.}
Let $t_1,\dots,t_n$ be terms. If $\Delta_G(t_1,\dots,t_n) = (u,(\overline{s_1},\dots,\overline{s_p}))$, then $t_i$, $t_i = u[\alpha_1\backslash s_{1,i},\dots,\alpha_p\backslash s_{p,i}]$ (for $1\leq i \leq n)$.
\label{thm:DeltaGSoundness}
\end{theorem}

\begin{proof}
W.l.o.g. we assume that the set of variables which occur in $u$ is $\{\alpha_1,\dots,\alpha_k\}$.\\
We proceed by induction on the depth $u$.

\paragraph{Base case.} $u$'s depth is 0.\\
If not all $t_1,\dots,t_n$ are equal, $\Delta_G(t_1,\dots,t_n) = (\alpha_1,\overline{s_1} = (t_1,\dots,t_n))$ per definition.
$t_i = u[\alpha_1\backslash s_{1,i},\dots,\alpha_p\backslash s_{p,i}] = \alpha_1[\alpha_1\backslash s_{1,i}] = t_i$.\\
If all terms are equal, then $\Delta_G(t_1,\dots,t_n) = (t_1,())$. We simply get $t_i = t_1$.

\paragraph{Step case.} Assume the soundness for a depth of $\leq d$. We show the soundess for a depth of $d+1$.\\

\noindent
If all terms are equal, $\Delta_G(t_1,\dots,t_n) = (t_1,())$, which is obviously correct.
If not all terms are equal, $\Delta_G(t_1,\dots,t_n) = (f(u_1,\dots,u_m), (\overline{s_1},\dots,\overline{s_p}))$.\\

\noindent
For the second case, let $\Delta_G(t_j^1,\dots,t_j^n) = (u_j,(\overline{s_{\alpha_1}}\dots,\overline{s_{\alpha_k}}))$. Per the IH, the soundness condition holds for $\Delta_G(t_j^1,\dots,t_j^n)\ (1\leq j\leq m)$ --- i.e. $t_j^i = u_j[\alpha_1\backslash s_{\alpha_1,i},\dots,\alpha_k\backslash s_{\alpha_k,i}]$. Per the uniqueness constraint on instances of $\alpha$, $u_{j_1}$ and $u_{j_2}$ will contain non-intersecting sets of $\alpha$ iff $j_1 \neq j_2$. If we therefore take $f(u_1,\dots,u_m)$ and the concatenation of all $\pi_2(\Delta_G(t_j^1,\dots,t_j^n))$, the soundness condition will still hold.\\

It may, however, be the case that for two different $\alpha_{j_1}$ and $\alpha_{j_2}$ ($j_1 < j_2$), $\overline{s_{j_1}} = \overline{s_{j_2}}$. In this case, \texttt{elim} replaces $\alpha_{j_2}$ with $\alpha_{j_1}$ in $u_1,\dots,u_m$ and deletes $\overline{s_{j_2}}$, effectively merging $\alpha_{j_1}$ and $\alpha_{j_2}$. For all $j_3 > j_2$, $\alpha_{j_3}$ is renamed to $\alpha_{j_3 - 1}$, preserving the 1-to-1 correspondence between $\alpha_i$ and $\overline{s_j}$ for all $j \in \{1,\dots,n-1\}$. Since only the superfluous $\alpha_{j_2}$ was eliminated, the substituiton $u[\alpha_1\backslash s_{1,i},\dots,\alpha_p\backslash s_{p,i}]$ (for any $i$) yields the same result as before and the soundness condition is still fulfilled.\\

\texttt{nub} then merely repeats this soundness-preserving operation until no more duplicates can be eliminated.

\end{proof}



\section{Completeness}

To define the sense in which $\Delta_G$ is complete, we must first introduce the notion of a normal form for substitutions. For a single $\alpha$, this was done in \cite[Ch. 4]{Hetzl2012}, where a calculus for decompositions into $u,S$ was presented. The full calculus, as well as the theoretical results of that paper will not be replicated for the case of multiple $\alpha$ here, but we will make use of a few analogous notions.

\subsection{A calculus of substitutions for multiple $\alpha$}

\begin{definition}
  \textbf{Substitution.}
  A term $u$ and a list of vectors $S$ are a substitution for a set of terms $\{t_1,\dots,t_n\}$ if
  $$
    \{t_1,\dots,t_n\} =
    u \circ S =
    \{u[\overline{\alpha}\backslash(s_{1,1},\dots,s_{q,1}),\dots,\overline{\alpha}\backslash(s_{1,n},\dots,s_{q,n})]\}
  $$
  where $\overline{\alpha} = (\alpha_1,\dots,\alpha_q)$, $u[\overline{\alpha}\backslash(s_{1,j},\dots,s_{q,j})] = u[\alpha_1\backslash s_{1,j},\dots,\alpha_q\backslash s_{q,j}]$, s.t.
  \begin{enumerate}
    \item $S$ does not contain any $\alpha_i$ and
    \item the variables occurring in $u$ are numbered $\alpha_1,\dots,\alpha_q$ left-to-right.
  \end{enumerate}
\end{definition}

\begin{definition} \textbf{Left-shifting.}
  A substitution may be left-shifted if, for some $\alpha_i$, all terms in the list $s_i$ start with a common function symbol $f$ of arity $r$. Let $s_i = (f(a_{1,1},\dots,a_{1,r}),\dots,f(a_{n,1},\dots,a_{n,r}))$. Then, left-shifting for $\alpha_i$ is defined as:

  \begin{prooftree}
    \AxiomCm{u\circ S}
    \RightLabelm{\leftarrow}
    \UnaryInfCm{u[\alpha_i\backslash f(\alpha_i^1,\dots,\alpha_i^r)] \circ S[\overline{s_i}\backslash (a_{1,1},\dots,a_{n,1}),\dots,(a_{1,r},\dots,a_{n,r}))]}
  \end{prooftree}

  \noindent
  where $\alpha_i^1,\dots,\alpha_i^r$ are fresh instances of $\alpha$.

  \paragraph{Example.} Let $\{t_1,\dots,t_n\} = \{f(g(a,b),x),f(g(c,d),y)\}$ and let\\
  $u = f(\alpha_1,\alpha_3),\ S=(\overline{s_1},\overline{s_2})$ with $\overline{s_1} = (g(a,b),g(c,d),\ \overline{s_2} = (x,y)$.\\
  We can left-shift for $\alpha_1$:

  \begin{prooftree}
    \AxiomCm{f(\alpha_1,\alpha_3) \circ (\overline{s_1},\overline{s_2})}
    \RightLabelm{\leftarrow}
    \UnaryInfCm{f(g(\alpha_1,\alpha_2),\alpha_3) \circ ((a,c),(b,d),(x,y))}
  \end{prooftree}
\end{definition}

\begin{definition}
  \textbf{Merging \& splitting $\alpha$.}
  Suppose that there exist $\alpha_{i}$ and $\alpha_{j}$ in $u$ ($i \neq j$) s.t. $\overline{s_i} = \overline{s_j}$. Then we can replace $\alpha_{j}$ with $\alpha_{i}$ and $\alpha_k$ with $\alpha_{k-1}\ (k > max{i,j})$ in $u$ and delete $\overline{s_{j}}$ from $S$. Conversely, we can rename multiple occurrences of $\alpha_i$ to $\alpha_{i_1},\dots,\alpha_{i_n}$ and duplicate $\overline{s_{i}}$ $n$ times. These operations are called merging \& splitting and obviously preserve the result of $u \circ S$.
\end{definition}

\begin{definition}
  \textbf{$\alpha$-elimination.} Suppose that, for a substitution $u,S$, there exists an $a_i$ in $u$ s.t. all terms in $\overline{s_i}$ are equal, i.e. $\overline{s_i} = (s_{i,1},\dots,s_{i,1})$. Then we can eliminate $\alpha_i$ by replacing $u$ with $u[\alpha_i\backslash s_{i,1}]$ and deleting $\overline{s_i}$ from $S$.
\end{definition}


\begin{definition} \textbf{Normal form.} A substitution $u,S$ is in normal form iff no left-shift, no merging and no $\alpha$-elimination are possible.
\end{definition}

\begin{theorem}Every substitution $u,S$ has a unique normal form. \label{thm:uniqueNormalForm}\end{theorem}

\begin{proof}
  \begin{enumerate}
    \item Observe that the order in which merging operations and $\alpha$-eliminations are applied is irrelevant. 
    \item The order in which two left-shift operations are applied is irrelevant. Suppose $\alpha_{j_1}$ and $\alpha_{j_2}$ occur in $u$ and
          we can left-shift for both. Left-shifting for $\alpha_{j_1}$ only replaces $\alpha_{j_1}$ in $u$ with a term containing fresh
          $\alpha_{j_1}^1,\dots,\alpha_{j_1}^r$ and the vector $\overline{s_{j_1}}$ in $S$ with new vectors $\overline{s_{j_1}^1},\dots,\overline{s_{j_1}^r}$.
          The applicability of a left-shift for $\alpha_{j_2}$ remains unaffected.
    \item Left-shifting and $\alpha$-elimination are confluent.
          Suppose both left-shifting and $\alpha$-elimination can be performed on $\alpha_j$, i.e.\\
          $\overline{s_j} = (f(s_{j,1},\dots,s_{j,r}),\dots,f(s_{j,1},\dots,s_{j,r}))$.
          $\alpha$-elimination replaces $\alpha_j$ with $f(s_{j,1},\dots,s_{j,r})$ in $u$ and deletes $\overline{s_j}$ in $S$.
          Left-shifting replaces $\alpha_j$ with $f(\alpha_j^1,\dots,\alpha_j^r)$ in $u$ and $\overline{s_j}$ with $(s_{j,1},\dots,s_{j,1})\dots,(s_{j,r},\dots,s_{j,r})$.
          $\alpha$-elimination can then be performed on $\alpha_j^1,\dots,\alpha_j^r$, giving the same result.
    \item Left-shifting and merging are confluent. Suppose that $\alpha_{j_1}$ and $\alpha_{j_2}$ occur in $u$ s.t. $\alpha_{j_1}$ and $\alpha_{j_2}$
          can be merged and left-shifting is possible on both.
          Since merging is possible, $\overline{s_{j_1}} = \overline{s_{j_2}} = (s_{j,1},\dots,s_{j,n})$. Merging $\alpha_{j_1}$ and $\alpha_{j_2}$ replaces $u$
          with $u[\alpha_{j_2}\backslash \alpha_{j_1}]$ and deletes $\overline{s_{j_2}}$ from $S$.
          If we left-shift on $\alpha_{j_1}$ and $\alpha_{j_2}$, we introduce $\alpha_{j_1}^1,\dots,\alpha_{j_1}^r$ and $\alpha_{j_2}^1,\dots,\alpha_{j_2}^r$
          with corresponding $s$-vectors $\overline{s_{j_1}^1},\dots,\overline{s_{j_1}^r}$ and $\overline{s_{j_2}^1},\dots,\overline{s_{j_2}^r}$
          s.t. $\overline{s_{j_1}^i} = \overline{s_{j_2}^i}\ (1 \leq i \leq r)$.
          Merging $(\alpha_{j_1}^1, \alpha_{j_2}^1), \dots, (\alpha_{j_1}^r, \alpha_{j_2}^r)$ then delivers the same result as merging $(\alpha_{j_1},\alpha_{j_2})$ did. 
  \end{enumerate}
\end{proof}

\begin{corollary}
If $t_1 = \dots = t_n$ and $n > 0$, the normal form of a substitution for $(t_1,\dots,t_n)$ is $t_1,()$.
\label{cor:allTermsEq}
\end{corollary}

\begin{proof}
Take the trivial substitution $\alpha_1,(t_1,\dots,t_n)$ and perform $\alpha$-elimination.
\end{proof}

%\begin{corollary}
%If $u,S$ is a substitution in which all possible left-shifts, but no merges and $\alpha$-eliminations, have been performed, its normal form $u',S'$ is the result of performing %every possible merge and $\alpha$-elimination.
%\label{cor:leftShiftsFirst}
%\end{corollary}

%\begin{proof}
%Follows from the fact that the order in which left-shifts, merges and $\alpha$-elimination are performed is irrelevant.
%\end{proof}

\medskip

%-----------------------------------------------------------------------------
% Completeness proof begins here
%-----------------------------------------------------------------------------

\subsection{Completeness proof}

\begin{lemma}
\textbf{\texttt{nub} performs all merges.}
W.l.o.g. let $u,S = (\overline{s_1},\dots,\overline{s_q})$ be a substitution in which the variables $\alpha_1,\dots,\alpha_q$ occur. Then $\texttt{nub}(u,S) = u',S'$ such that $u',S'$ equals $u,S$ modulo merging and that no merge is possible in $u',S'$.
\label{thm:nubAllMerges}
\end{lemma}

\begin{proof}
We show that if a merge is possible, \texttt{elim} performs it.\\

\noindent
Per definition, merging two distinct variables $\alpha_i, \alpha_j$ is possible iff $\overline{s_i} = \overline{s_j}$. This is equivalent to the condition for the application of \texttt{elim}:\\ $[\exists i,j: i < j]\ \overline{s_{i}} = \overline{s_{j}}$.
If this condition holds, \texttt{elim} performs the two actions given in the definition of merging: it
  \begin{enumerate}
    \item removes $\overline{s_j}$ and
    \item replaces $\alpha_j$ with $\alpha_i$ and $\alpha_k$ with $\alpha_{k-1}\ (k > j)$ in $u$.
  \end{enumerate}

\texttt{nub} is the least fixed point of \texttt{elim}, and since \texttt{elim} merges two variables $\alpha_i$ and $\alpha_j$ iff they can be merged, it is shown that \texttt{nub} performs exactly every possible merge in $u,S$.
\end{proof}

\begin{theorem}
  \textbf{Completeness of $\Delta_G$.}
  If $u,S$ is a substitution for a non-empty set of terms $\{t_1,\dots,t_n\}$, then $\Delta_G(t_1,\dots,t_n) = (u',S')$ s.t. $u',S'$ is the normal form of $u,S$.
  \label{thm:DeltaGCompleteness}
\end{theorem}





\begin{proof}
  W.l.o.g. we assume a contiguous numbering of the $\alpha$ occurring in $u$.
  We proceed by induction on the depth of $t_1,\dots,t_n$.

  \paragraph{Base case.} Let $t_i\ (1 \leq i \leq n)$ have a depth of 0. We have two sub-cases:
  \begin{enumerate}
    \item All terms are equal. This is the condition of first case of $\Delta_G$, which, per Corollary~\ref{cor:allTermsEq}, is the normal form of $u,S$.
    \item Not all terms are equal. The terms can only be variables or constants. In either case, the only possibly substitution is $u=\alpha, S=(t_1,\dots,t_n)$. This is exactly the ``otherwise''-case of $\Delta_G$.
  \end{enumerate}

  \paragraph{Step case.} Assume that $\Delta_G$ is complete if $t_i\ (1 \leq i \leq n)$ has a depth $\leq d$.
  We again have two cases:
  \begin{enumerate}
    \item All terms are equal. As in the base case, $\Delta_G$ computes the normal form of $u,S$, per Corollary~\ref{cor:allTermsEq}.
    \item Not all terms are equal. If not all terms begin with a common function symbol with arity $m$, the only possible substitution is $u=\alpha, S=(t_1,\dots,t_n)$. This is again exactly the ``otherwise''-case of $\Delta_G$.
    If all terms do begin with a common function symbol of arity $m$, then $t_i = f(t_1^i,\dots,t_m^i)$ for  $(1 \leq i \leq n)$. This is the condition for the second case of $\Delta_G$. $u$ can take one of two forms:
    \begin{enumerate}
      \item $u=\alpha$. Since a left-shift is possible due to the common function symbol $f$, such a substitution is not in normal form. By left-shifting, we arrive at the second case.
      \item $u=f(u_1,\dots,u_m)$. In this case, $u_i, S_i\ (1 \leq i \leq m)$ must be a substitution for the terms $t_i^1,\dots,t_i^n$. For these substitutions, completeness of $\Delta_G$ holds per the IH. Now let $\widehat{u}=f(\widehat{u_1},\dots,\widehat{u_m}),\widehat{S}=\bigsqcup\limits_{1 \leq i \leq m} S_i$ be the substitution created by simply concatenating all $u_i,S_i$, where $\widehat{u_i}$ is $u_i$, with its variables renamed to avoid clashes between any two $u_{i_1}, u_{i_2}$. Per definition of $\Delta_G$, $\widehat{u},\widehat{S}$ is precisely the result of the second case of $\Delta_G$, without the application of \texttt{nub}. Furthermore, per Theorem~\ref{thm:uniqueNormalForm}, $u,S$ and $\widehat{u},\widehat{S}$ have the same normal form.\\

      Due to the IH, $\widehat{u},\widehat{S}$ can only differ from $u',S'$ through merging. Per Lemma~\ref{thm:nubAllMerges}, the application of \texttt{nub} to $\widehat{u},\widehat{S}$ performs all possible merges, which results in the unique normal form of both $\widehat{u},\widehat{S}$ and $u,S$.
    \end{enumerate}
  \end{enumerate}
\end{proof}

\begin{corollary}
  Every non-empty set of terms $\{t_1,\dots,t_n\}$ has exactly one substitution in normal form and $\Delta_G$ computes it.
  \label{thm:deltaGCorrectness}
\end{corollary}

\begin{proof}
  Follows from Theorems~\ref{thm:DeltaGSoundness}, \ref{thm:uniqueNormalForm} \& \ref{thm:DeltaGCompleteness} and from $\Delta_G$ being a total function.
\end{proof}


\section{$\Delta$-Table}

The $\Delta$-table, as described in the paper, is compatible with the $\Delta_G$-vector and, save for substituting for multiple $\alpha$ instead of one, can be left as is. 

\section{Remarks}

The introduction of a global, contiguous numbering of $\alpha$-instances is very procedural and bloats the definition almost to the point of being pseudocode, but the precise definition of the semantics of $\Delta_G$ and the continuous \& unique labeling and and re-labeling of $\alpha$-instances regrettably make such an algorithmic approach necessary.\\

\section{Extensions}

For theoretical reasons, we might be interested in limiting the number of allowed $\alpha$ (bounded $\Delta_G$).

\subsection{Bounded generalized $\Delta$-Vector}

$\Delta_G$ will compute a unique decomposition that will preserve as much of the common structure of $t_1,\dots,t_n$ as possible, employing as many $\alpha$-instances as needed, but in some cases, it may be desirable to limit the number of such $\alpha$-instances that it may use.
Consider the following example:

$$
\begin{array}{l}
\Delta_G(f(g(a,b),g(c,d),e), f(g(x,y),g(u,v),w)) =\\
\\
(f(g(\alpha_1,\alpha_2),g(\alpha_3,\alpha_4),\alpha_5); (a,x), (b,y), (c,u), (d,v), (e,w))
\end{array}
$$

This illustrates two points:
\begin{enumerate}
\item There is a non-deterministic choice: we could restrict the number of $\alpha$-instances to, say, 4. This can be achieved by right-shifting $\alpha_1$ and $\alpha_2$ into a new $\alpha'$ (with the terms $(g(a,b),g(x,y))$) or $\alpha_3$ \& $\alpha_4$ (with the terms $(g(c,d),g(u,v))$).
\item Generally, it is not possible to specify exactly how many $\alpha$ should occur in $u$: if we restrict the number of $\alpha$-instances to at most 2, we only get the trivial decomposition $(\alpha; f(g(a,b),g(c,d),e), f(g(x,y),g(u,v),w))$.
\end{enumerate}

We therefore can specify upper bounds, but not lower bounds on the number of $\alpha$-instances which may be used, although this burdens us with a non-deterministic choice as to for which $\alpha$-instances to right-shift.\\

\subsubsection{Calculating the set of all substitutions with at most $k$ variables}

$\Delta_G$ can be slightly altered to compute substitutions with at most $k$ variables. We do this by introducing a parameter $k$ and a non-deterministic boolean variable \textbf{RANDOM}, creating $\Delta_G^k$:

$$
  \begin{array}{l l l}
    \Delta_G^*(t_1,\dots,t_n) & = & \left\{
      \begin{array}{l}

      (t_1,()) \mt{ if } t_1 = t_2 = \dots = t_n \mt{ and } n > 0\\
      \\

      \texttt{nub}(f(u_1,\dots,u_m), (\overline{s_1},\dots,\overline{s_q}))
      \\
      \quad
      \begin{array}{l l}
        \mt{if all } & t_i = f(t_1^i,\dots,t_m^i) \mt{, case 1 does not apply and}\\
        & (\mt{\textbf{RANDOM} = true} \vee m = 1)\\
        % & [\nexists l]\ l > k \wedge \alpha_l \mt{ occurs in } f(u_1,\dots,u_m),\\
        \mt{where } & (\overline{s_1},\dots,\overline{s_q}) = \bigsqcup\limits_{1 \leq j \leq m} \pi_2(\Delta_G(t_j^1,\dots,t_j^n)) \mt{ and}\\
                    & u_j = \pi_1(\Delta_G(t_j^1,\dots,t_j^n)) \mt{ for all } j \in \{1,\dots,m \}\\
      \end{array}
      
      \\
      \\
      
      (\alpha_{\mt{UNIQUE}},(t_1,\dots,t_n))\quad \mt{otherwise}\\
      \end{array}
    \right.

    \\
    \\

      \Delta_G^k(t_1,\dots,t_n) & = & \{u,S \in \Delta_G^*(t_1,\dots,t_n) |\ [\nexists l]\ l > k \wedge \alpha_l \mt{ occurs in } u\}
    \end{array}
$$

The side-condition for the instantation of $\alpha_{\mt{UNIQUE}}$ is the same as in $\Delta_G$: the leftmost occurrence is $\alpha_1$ and all instances are numbered incrementally left-to-right.\\

The second case of $\Delta_G^*$ now has an condition: either \textbf{RANDOM} is true or the function symbol $f$ at the head of $t_1,\dots,t_n$ is unary. This for the non-deterministic nature of right-shifting: whenever \textbf{RANDOM} is false and choosing the second case might introduce more than variable, we terminate our search for a common structure in the termset early and create a new variable with a corresponding $s$-vector consisting of $t_1,\dots,t_n$, which is equivalent to computing $\Delta_G$ and the non-deterministically right-shifting to reduce the number of variables. $\Delta_G^k$ simply filters the result of $\Delta_G^*$ so that we only get substitutions with at most $k$ variables.\\

It is obvious, substitutions computed by $\Delta_G^k$ need not be in normal form. To talk about the result of $\Delta_G^k$, we thus need the notion of a weaker normal form for substitutions:

\begin{definition}
\textbf{Weak k-normal form.}
A substitution $u,S$ is in weak k-normal form if $\leq k$ variables occur in $u$, merging and $\alpha$-elimination are not possible, and any left-shifting would result in $> k$ variables in $u$.
\end{definition}

\noindent
Thus equipped, we can give the following completeness result:

%SOUNDNESS:
%The algorithm isn't sound!
%Randomly jumping into the otherwise-case is premature. Even if we would temporarily get too many
%variables, we still might be able to merge them back!

%I don't think this can be easily fixed; calculating Delta_G and right-shifting might prove more fruitful, both in theory and practice.

%\begin{theorem}
%\textbf{Soundness of $\Delta_G^k$.}
%Let $t_1,\dots,t_n$ be terms. If $u,S \in \Delta_G^k(t_1,\dots,t_n)$, then $u,S$ is in weak l-normal form ($l \leq k$) and $t_i = u[\alpha_1\backslash s_{1,i},\dots,\alpha_p\backslash s_{p,i}]$ (for $1\leq i \leq n)$.
%\label{thm:DeltaGkSoundness}
%\end{theorem}

%\begin{proof}
%The proof is largely analogous to that given in Theorem~\ref{thm:DeltaGSoundness}. Since $\Delta_G^k$ only differs from $\Delta_G$ in that it non-deterministically chooses its ``otherwise''-case, we need only show two additional claims:

%\begin{enumerate}
%  \item $u,S$ is in weak $l$-normal form for some $l < k$ and
%  \item choosing the ``othwise''-case where $\Delta_G$ would choose its second case still results in a substitution.
%\end{enumerate}

%The first claim is clearly true: since $\Delta_G^k$ filters out all substitutions returned by $\Delta_G^*$ which have more than $k$ variables, $\leq k$ variables occur in $u$. If left-shifting is not possible, $u,S$ is weak $k$-normal form. Otherwise, suppose that left-shifting is possible for some variable $\alpha_i$ and a corresponding s-vector $(f(\dots),\dots,f(\dots))$. $f$ cannot be unary, since for unary functions, the second case of $\Delta_G^*$ would have applied. Left-shifting for $\alpha_i$ would therefore introduce at least two new variables $\alpha_{i_1}$ and $\alpha_{i_2}$. It is not possible to merge these two variables with each other or with existing ones, because the application

% in the second case of $\Delta_G^k$: should a variable with an index $ > k$ occur, the ``otherwise''-case is chosen, which instantates one variable and creates a substitution . The second claim is similarly clear: if we choose the ``otherwise''-case and instantiate $\alpha_i$ with the s-vector $(t_1,\dots,t_n)$,\\
%$\alpha_i\backslash(t_1,\dots,t_n)=(t_1,\dots,t_n)$.
%\end{proof}

%-------------------------------------------------------------------------------------------------

\begin{theorem}
\textbf{Completeness of $\Delta_G^k$.}
If we evaluate for all possible assignments for \textbf{RANDOM}, $\Delta_G^k$ returns the set of all substitutions which are in weak l-normal form ($1 \leq l \leq k$).
\label{thm:DeltaGkCompletenss}
\end{theorem}

\begin{proof}
Let $l \in \{1,\dots,k\}$ and let $u,S$ be a substitution in weak l-normal form. We proceed by induction on the depth of $t_1,\dots,t_n$.
The proof is entirely analogous to that for Theorem~\ref{thm:DeltaGCompleteness}, except for point 2 of the step case.
There, if $t_1\dots,t_n$ start with a common unary function symbol, $u$ can only be of the form $u=f(u_1)$, since it wouldn't be in weak l-normal form otherwise. This corresponds to second case of $\Delta_G^*$, where the condition $(\textbf{RANDOM = true } \vee m=1)$ becomes $\top$ and thereby identical to the condition for the second case of $\Delta_G$.
If, on the other hand, $t_1\dots,t_n$ start with a non-unary function symbol of arity $m$, $u$ may either be of the form $u=\alpha$ or $u=f(u_1,\dots,u_m)$. The first form corresponds to the ``otherwise''-case and the second form corresponds to the second case of $\Delta_G^k$. Due to \textbf{RANDOM}, both forms are chosen and hence $\Delta_G^k$ is complete.
\end{proof}

\medskip

Unlike the normal form, weak k-normal forms are not confluent, as the following example shows:

$$
\begin{array}{l}
\Delta_G^\infty(f(g(a,b),g(c,d),e), f(g(x,y),g(u,v),w)) =\\
\\
\{(f(\alpha_1,\alpha_2,\alpha_3); ((g(a,b),g(x,y)), (g(c,d),g(u,v)), (e,w)),\dots \}
\end{array}
$$

As we can see, this is the same termset as in the previous example, but here, we have terminated the search early in the case of $\alpha_1$ and $\alpha_2$. The substitution shown has 3 variables. If we want to get to weak 4-normal form, we can left-shift either for $\alpha_1$ and $\alpha_2$, but not for both. Therefore, we can create two different weak normal forms.

\bibliographystyle{plain}
\bibliography{references}
\end{document}