\documentclass[a4paper,11pt]{article}
\usepackage[utf8]{inputenc}
%\usepackage{fullpage}
\usepackage{listings}
\usepackage{color}
\usepackage{url}
\usepackage{alltt}
\usepackage{multicol}
\usepackage{fancyhdr}
\usepackage{amsmath}
\usepackage{bussproofs}

\renewcommand{\familydefault}{\sfdefault} % change to sans serif font

\newcommand{\seq}{\vdash}	% the sequent sign
\newcommand{\impl}{\supset} %logical connectives: implies, not, and, or
\renewcommand{\lnot}{\neg}
\renewcommand{\land}{\wedge}
\renewcommand{\lor}{\vee}

%Commands for constructing proof trees with bussproofs. See the chapter on the LK system for examples.
\newcommand{\UnaryInfCm}[1]{\UnaryInfC{$#1$}}
\newcommand{\BinaryInfCm}[1]{\BinaryInfC{$#1$}}
\newcommand{\RightLabelm}[1]{\RightLabel{$#1$}}
\newcommand{\AxiomCm}[1]{\AxiomC{$#1$}}

% Normal text in math mode ("math text")
\newcommand{\mt}[1]{\textnormal{#1}}
% CLI-style names,words,... within normal text
\newcommand{\cli}[1]{{\tt {#1}}}

\newenvironment{meta}{\color{red}}{\color{black}}

\definecolor{LightGray}{rgb}{0.9,0.9,0.9}
\lstset{
  backgroundcolor=\color{LightGray},
  basicstyle=\footnotesize\ttfamily,
  breaklines=true,
  frame=single,
  mathescape=true,
  inputencoding=utf8,
  extendedchars=true,
  literate={∧}{{\ensuremath{\land}}}1 {⊃}{{\ensuremath{\impl}}}1 {∨}{{\ensuremath{\lor}}}1 {¬}{{\ensuremath{\neg}}}1 {∀}{{\ensuremath{\forall}}}1 {∃}{{\ensuremath{\exists}}}1 {ο} {{\ensuremath{o}}}1 {ι}{{\ensuremath{\iota}}}1
}

\setlength{\parindent}{0pt}
\setlength{\parskip}{4pt}
\setlength{\headheight}{14pt}
\setlength{\oddsidemargin}{1pt}
\setlength{\textwidth}{450pt}
%\setlength{\textheight}{600pt}

\pagestyle{fancy}
\lhead{GAPT -- User Manual}
\chead{}
\rhead{}

\begin{document}

\begin{titlepage}
\begin{center}

\hrule

\vspace*{20mm}

{\Huge GAPT}

\vspace*{5mm}

{\huge Generic Architecture for Proof Transformations}

\vspace*{20mm}

{\Huge User Manual}

\vspace*{50mm}

{\Large \today}

\vspace*{20mm}

{\Large
% Whenever you contribute for this manual, include your name here
% (alphabetically by last name) :)
Stefan Hetzl - \texttt{stefan.hetzl@tuwien.ac.at}\\
Bernhard Mallinger - \texttt{bernhard@logic.at}\\
Giselle Reis - \texttt{giselle@logic.at}\\
Janos Tapolczai - \texttt{e0825077@student.tuwien.ac.at}\\
Daniel Weller - \texttt{weller@logic.at}
}

\vspace*{20mm}

\hrule
\end{center}



\end{titlepage}

\tableofcontents
\vfill
\pagebreak

\section{Introduction}

GAPT is a generic architecture for proof transformations. It is implemented
in Scala and its logical basis is simple type theory. It has
layers for first-order logic and schematic first-order logic.

The focus of GAPT are proof transformations (in contrast to proof assistants
whose focus is proof formalization and automated deduction systems whose focus
is proof search). GAPT is used from a shell which provides access to the functionality
in the system in a way that is inspired by computer algebra systems: the basic
objects are formulas and (different kinds of) proofs which can be modified
by calling GAPT-commands from the command-line. In addition, there
is a graphical user interface which allows to view (and up to a certain extent:
also to modify) proofs in a flexible and visually appealing way.

The current functionality of GAPT includes data structures for formulas,
sequents, resolution proofs, sequent calculus proofs, expansion tree proofs
and algorithms for e.g.\ unification, proof skolemization, cut-elimination,
cut-elimination by resolution~\cite{Baaz00CutElimination}, cut-introducton~\cite{Hetzl2012}, etc.

This document describes the system from the perspective of a user who has
downloaded a jar-file. For information on how to get started as a developer,
please see the developer wiki at \url{http://code.google.com/p/gapt/w/list}.

\begin{meta}
Meta-comments about this user guide are printed like this.
\end{meta}

\section{System Requirements}
\label{sec:sysreq}

To run GAPT you need to have the Java runtime environment 1.6.0 or higher installed. We had
problems with OpenJDK, but you can use it at your own risk.

GAPT contains interfaces to the following automated reasoning systems. Installing
them is optional. If GAPT does not find the executables in the path, the
functionality of these systems will not be available. 
%
\begin{itemize}
\item Prover9 (\url{http://www.cs.unm.edu/~mccune/mace4/download/}) - make sure
the commands \texttt{prover9}, \texttt{prooftrans} and \texttt{tptp\_to\_ladr}
are available.
\item VeriT (\url{http://www.verit-solver.org/})
\item Minisat (\url{http://minisat.se/})
\end{itemize}

\section{Downloading and Running the System}

You can download a pre-packaged jar-file of the current version of GAPT
at~\url{http://code.google.com/p/gapt/downloads/list}. After having downloaded
the gapt-cli zip-file you will find a shellscript \texttt{cli.sh}.

Running this script
\begin{lstlisting}[mathescape=false]
$ ./cli.sh
\end{lstlisting}
will start the command-line interface of GAPT.

The command-line interface of GAPT runs in an interactive Scala-shell. This
has the consequence that all functionality of Scala is available to you. In
particular it is easy to write Scala-scripts that use the functionality of GAPT.

Here are some useful things you should know about the Scala shell.

If you want to assign the result of a method to a variable, use: 
\texttt{var v = method(args)}. Otherwise, the system will create a variable by 
itself. You can see it's name and type before the description, like this:

With a variable name:

\begin{lstlisting}
scala> var i = 12
i: Int = 12
\end{lstlisting}

Without a variable name (in this case, the system created the variable \texttt{res12}):

\begin{lstlisting}
scala> 12
res12: Int = 12
\end{lstlisting}

To see the value of a variable, just type it’s name and press enter.

The elements of a list in Scala are indexed using parenthesis. So if \texttt{lst} 
is a list, the first element is obtained by \texttt{lst(0)}. It is also possible 
to use the methods \texttt{lst.head} (for the first element) and \texttt{lst.tail} 
(for the rest of the list).

The elements of a tuple \texttt{t} of size $n$ are accessed by the methods 
\texttt{t.\_1}, \texttt{t.\_2}, … , \texttt{t.\_n}.

To quit the interactive shell, just type \texttt{:quit} and press enter.

In order to see the list of all available commands, type \texttt{help} and
press enter.

\section{File Input/Output}\label{sec.fileio}

\begin{meta}
TODO: first some general talk on file formats, then: give examples for saving and loading
proofs.
\end{meta}

This method \texttt{loadProofs} will take as an argument a string that represents the path of a file
containing a proofdatabase in the xml format (generated by HLK for example), and will return a list 
of pairs. It expects a file (the string of a proof will not work) and you can use 
the relative path. On the list returned, each pair is composed of a string and 
an object representing a proof within the system. The string is the name of the 
proof defined on the xml file. For example:
%
\begin{lstlisting}
scala> val proofs = loadProofs( "../examples/simple/fol1.xml.gz" )
\end{lstlisting}
%
returns a list of length 1 as shown by entering
%
\begin{lstlisting}
scala> proofs.length
\end{lstlisting}
%
Its only element is \cli{proofs(0)}, the name of this proof can be obtained by
entering
%
\begin{lstlisting}
scala> proofs(0)._1
\end{lstlisting}
%
and the proof itself by
%
\begin{lstlisting}
scala> val proof = proofs(0)._2
\end{lstlisting}
%
You can then view this proof in the graphical user interface prooftool by 
entering
%
\begin{lstlisting}
scala> prooftool( proof )
\end{lstlisting}
%
In the folder \cli{../examples/simple} you can find a number of further simple
examples that illustrate different aspects of GAPT.




%\paragraph{\textbf{loadProofDB: String $\rightarrow$ ProofDatabase}}
%{\color{red}TODO}

The command \texttt{exportXML: List[Proof], List[String], String $\rightarrow$ Unit}
Exports several proofs to an XML file. The first argument is the list of proofs,
the second is the list of the names of the proofs and the third is the name of
the file that will be written. The file path can also be specified and it's
relative to where the program is being executed.

The code below exports the proofs in the variables $p1$ and $p2$ to the file
\texttt{result.xml} with names ``First proof'' and ``Second proof'',
respectively.
\begin{lstlisting}
scala> exportXML(p1::p2::Nil, "First proof"::"Second proof"::Nil, "result.xml")
\end{lstlisting}

%\paragraph{\textbf{exportTPTP: List[Proof], List[String], String $\rightarrow$ Unit}}
%{\color{red}TODO}

\section{Entering Formulas}\label{sec.entering_formulas}

Formulas can be entered as strings which are parsed as follows:
\begin{lstlisting}
scala> val F = parse.fol( "Imp A Imp B And A B" )
F: at.logic.language.fol.FOLFormula = (A$\supset$(B$\supset$(A$\land$B)))
\end{lstlisting}
For a first-order examples consider:
\begin{lstlisting}
scala> val G = parse.fol( "Forall x Imp P(x,f(x)) Exists y P(x,y)" )
G: at.logic.language.fol.FOLFormula = $\forall$x.(P(x, f(x))$\supset\exists$y.P(x, y))
\end{lstlisting}

Also prover9 syntax is supported: %TODO: add citation

\begin{lstlisting}
scala>  val PG = parse p9 "(all x (P(x,f(x)) -> (exists y P(x,y))))"
PG: at.logic.language.fol.FOLFormula = ∀x.(P(x, f(x))⊃∃y.P(x, y))
\end{lstlisting}

The prover9 syntax was also extended to higher-order logic, where type declarations are added:

\begin{lstlisting}
scala> val HG = parse hlkformula "var P:o>i>o; const f:o>i; var x:o; var y:i; (all x (P(x,f(x))) -> (exists y P(x,y)))"
HG: at.logic.language.hol.HOLFormula = ∀x:ο.(P(x:ο, f(x:ο):ι): o⊃∃y:ι.P(x:ο, y:ι): o)
\end{lstlisting}


\begin{meta}
TODO: add the other parsers, describe syntax in detail
\end{meta}

A collection of formula sequences can be found in the file \cli{../examples/FormulaSequence.scala}.
Have a look at this code to see how to compose formulas without the parser. This file is
a scala script that can be loaded into the CLI by entering
%
\begin{lstlisting}
scala> :load ../examples/FormulaSequences.scala
\end{lstlisting}
%
Then you can generate instances of these formula sequences by entering e.g.
%
\begin{lstlisting}
scala> val f = BussTautology( 5 )
\end{lstlisting}



\section{Automated Deduction}
  
\subsection{SAT Solving using MiniSAT}
%
The following shows an example session, using the MiniSAT solver
to verify valdity and satisfiability, and query the thus obtained models.
Consider the {\em pigeon hole principle for $(m, n)$, $\mathrm{PHP}_{m,n}$}, which states that if $m$ pigeons
are put into $n$ holes, then there is a hole which contains two pigeons. It is valid
iff $m>n$. $\neg\mathrm{PHP}_{m,n}$ states that when putting $m$ pigeons into $n$ holes, there
is no hole containing two pigeons. This is satisfiable iff $m\leq n$. Make sure
that the pigeon hole principle is available in your current CLI session 
by entering
\begin{lstlisting}
scala> :load ../examples/FormulaSequences.scala
\end{lstlisting}
if you have not done so already. Then
\begin{lstlisting}
scala> miniSATprove(PigeonHolePrinciple(3, 2))
res12: Boolean = true
\end{lstlisting}
shows that $\mathrm{PHP}_{3,2}$ is valid, and
\begin{lstlisting}
scala> miniSATprove(PigeonHolePrinciple(3, 3))
res13: Boolean = false
\end{lstlisting}
shows that $\mathrm{PHP}_{3,3}$ is not valid.
Furthermore,
\begin{lstlisting}
scala> val m = miniSATsolve(Neg(PigeonHolePrinciple(3, 3))).get
\end{lstlisting}
yields a model of $\neg\mathrm{PHP}_{3,3}$ that can be queried:
\begin{lstlisting}
scala> val p1 = at.logic.testing.PigeonHolePrinciple.atom(1, 1)
p1: at.logic.language.fol.FOLFormula = R(p_1, h_1)

scala> val p2 = at.logic.testing.PigeonHolePrinciple.atom(2, 1)
p2: at.logic.language.fol.FOLFormula = R(p_2, h_1)

scala> val p3 = at.logic.testing.PigeonHolePrinciple.atom(3, 1)
p3: at.logic.language.fol.FOLFormula = R(p_3, h_1)

scala> m.interpret(p1) // Is pigeon 1 in hole 1?
res14: Boolean = false

scala> m.interpret(p2) // Is pigeon 2 in hole 1?
res15: Boolean = true

scala> m.interpret(p3) // Is pigeon 3 in hole 1?
res16: Boolean = false
\end{lstlisting}
We can also interpret quantifier-free formulas:
\begin{lstlisting}
cala> m.interpret( And(p1, p2) )
res17: Boolean = false

scala> m.interpret( Neg(notvalid) )
res18: Boolean = true
\end{lstlisting}

\subsection{SMT Solving using VeriT}


\subsection{First-Order Resolution Proving using Prover9}

\subsection{Built-In Resolution Prover}

GAPT contains a built-in resultion prover that can be called with the command:
\texttt{refuteFOL: Seq[Clause] $\rightarrow$ Option[ResolutionProof[Clause]]}
and with the command
\texttt{refuteFOLI: Seq[Clause] $\rightarrow$ Option[ResolutionProof[Clause]]}
for interactive mode.

\subsection{Built-In Tableaux Prover}

GAPT contains a  built-in tableaux prover for propositional logic
which can be called with the command \texttt{proveProp}, for example as in:
\begin{lstlisting}
scala> proveProp( parse.fol( "Imp A Imp B And A B" ))
\end{lstlisting}

\section{Entering Proofs}

There are various possibilities for entering proofs into the system. The most
basic one is a direct top-down proof-construction using the constructors
of the inference rules. For example, continuing in the environment of
Section~\ref{sec.entering_formulas}, suppose that we want to enter a
proof of \texttt{F}. It is convenient to prepare the subformulas first.
\begin{lstlisting}
scala> val F1 = parse.fol( "Imp B And A B" )
F1: at.logic.language.fol.FOLFormula = (B$\supset$(A$\land$B))

scala> val F2 = parse.fol( "And A B" )
F2: at.logic.language.fol.FOLFormula = (A$\land$B)

scala> val A = parse.fol( "A" )
A: at.logic.language.fol.FOLFormula = A

scala> val B = parse.fol( "B" )
B: at.logic.language.fol.FOLFormula = B
\end{lstlisting}
%
We start with the axioms:
%
\begin{lstlisting}
scala> val p1 = Axiom( A::Nil, A::Nil )
p1: at.logic.utils.ds.trees.LeafTree[at.logic.calculi.lk.base.Sequent] with at.logic.calculi.lk.base.NullaryLKProof{def rule: at.logic.calculi.lk.propositionalRules.InitialRuleType.type} = InitialRuleType(A :- A)

scala> val p2 = Axiom( B::Nil, B::Nil )
p2: at.logic.utils.ds.trees.LeafTree[at.logic.calculi.lk.base.Sequent] with at.logic.calculi.lk.base.NullaryLKProof{def rule: at.logic.calculi.lk.propositionalRules.InitialRuleType.type} = InitialRuleType(B :- B)
\end{lstlisting}
%
These are joined by an $\land:\mathrm{right}$-inference. See Appendix~\ref{app:sequent_calculus}
for the formal definition of the sequent calculus used in GAPT.
%
\begin{lstlisting}
scala> val p3 = AndRightRule( p1, p2, A, B )
p3: at.logic.calculi.lk.base.BinaryLKProof with at.logic.calculi.lk.base.BinaryLKProof with at.logic.calculi.lk.base.AuxiliaryFormulas with at.logic.calculi.lk.base.PrincipalFormulas = AndRightRuleType(A, B :- (A∧B), InitialRuleType(A :- A), InitialRuleType(B :- B))
\end{lstlisting}
%
To finish the proof it remains to apply two $\impl:\mathrm{right}$-inferences:
%
\begin{lstlisting}
scala> val p4 = ImpRightRule( p3, B, F2 )
p4: at.logic.utils.ds.trees.UnaryTree[at.logic.calculi.lk.base.Sequent] with at.logic.calculi.lk.base.UnaryLKProof with at.logic.calculi.lk.base.AuxiliaryFormulas with at.logic.calculi.lk.base.PrincipalFormulas = ImpRightRuleType(A :- (B⊃(A∧B)), AndRightRuleType(A, B :- (A∧B), InitialRuleType(A :- A), InitialRuleType(B :- B)))

scala> val p5 = ImpRightRule( p4, A, F1 )
p5: at.logic.utils.ds.trees.UnaryTree[at.logic.calculi.lk.base.Sequent] with at.logic.calculi.lk.base.UnaryLKProof with at.logic.calculi.lk.base.AuxiliaryFormulas with at.logic.calculi.lk.base.PrincipalFormulas = ImpRightRuleType( :- (A⊃(B⊃(A∧B))), ImpRightRuleType(A :- (B⊃(A∧B)), AndRightRuleType(A, B :- (A∧B), InitialRuleType(A :- A), InitialRuleType(B :- B))))
\end{lstlisting}
%
You can now view this proof by typing:
\begin{lstlisting}
scala> prooftool( p5 )
\end{lstlisting}

The system comes with a collection of example proof sequences in the file
\cli{../examples/ProofSequences.scala} which are generated in the above style.
Have a look at this code for more complicated proof constructions. In order
to load these proof sequences into the CLI, enter:
\begin{lstlisting}
scala> :load ../examples/ProofSequences.scala
\end{lstlisting}

\begin{meta}
TODO: mention shlk! and llk -- at least: give a short example for each
\end{meta}


\section{Proof Theory}

\subsection{Cut-Elimination (Gentzen's method)}

The GAPT-system contains an implementation of reductive cut-elimination
\`{a} la Gentzen. It can be used as follows: first we load a proof p
with cuts (as in Section~\ref{sec.fileio}).
%
\begin{lstlisting}
scala> val p = loadProofs( "../examples/simple/fol1.xml.gz" )(0)._2
\end{lstlisting}
%
and then call the cut-elimination procedure:
\begin{lstlisting}
scala> val q = eliminateCuts( p )
\end{lstlisting}



\subsection{Skolemization}

\begin{meta}
short example for proof skolemization 
\end{meta}


\subsection{Interpolation}

The command \texttt{extractInterpolant} extracts an interpolant from a cut-free
sequent calculus proof. The implementation is based on Lemma 6.5 of~\cite{Takeuti87Proof}. The method expects
a proof p and an arbitrary partition of the end-sequent $\Gamma \seq \Delta$ of p into a 
``negative part'' $\Gamma_1\seq\Delta_1$ and a ``positive part'' $\Gamma_2 \seq \Delta_2$.
It returns a formula $I$ s.t.\ $\Gamma_1\seq\Delta_1, I$ and $I,\Gamma_2\seq\Delta_2$
are provable and $I$ contains only such predicate symbols that appear in both, $\Gamma_1\seq\Delta_1$
and $\Gamma_2\seq\Delta_2$. For example, suppose pr is a proof of $p \lor q \seq p, q$
by a single $\lor$-left inference, then you can compute an interpolant as follows:
\begin{lstlisting}
scala> val s = pr.root
s: at.logic.calculi.lk.base.Sequent = (p:o $\lor$ q:o) :- p:o, q:o

scala> val npart = Set( s.antecedent( 0 ), s.succedent( 0 ) )
npart: scala.collection.immutable.Set[at.logic.calculi.occurrences.FormulaOccurrence] = Set((p:o $\lor$ q:o)[10005], p:o[10006])

scala> val ppart = Set( s.succedent( 1 ) )
ppart: scala.collection.immutable.Set[at.logic.calculi.occurrences.FormulaOccurrence] = Set(q:o[10007])

scala> val I = extractInterpolant( pr, npart, ppart )
I: at.logic.language.hol.HOLFormula = ($\bot$:o $\lor$ q:o)
\end{lstlisting}

\subsection{Expansion Trees}

Expansion trees are a compact representation of cut-free proofs. They have originally been
introduced in~\cite{Miller87Compact}. GAPT contains an implementation of
expansion trees for higher-order logic including functions for extracting expansion
trees from proofs, for merging expansion trees, for pruning and transforming them
in various ways and for viewing
them in a comfortable way in the graphical user interface.

An expansion tree contains the instances of the quantifiers for a formula. In order
to represent an LK-proof we use {\em expansion sequents}, i.e.~sequents of expansion trees.

Suppose we are in the command-line interface and have a value \cli{p} which is a cut-free \cli{LKProof}.
We can then extract the expansion sequent of \cli{p} by:
\begin{lstlisting}
scala> val E = extractExpansionTrees( p )
\end{lstlisting}
This expansion sequent can then be viewed in the graphical user interface by simply calling:
\begin{lstlisting}
scala> prooftool( E )
\end{lstlisting}
Prooftool is then intialized with displaying the end-sequent of \cli{p}, i.e.\ the shallow sequent
of \cli{E}. The user can then selectively expand quantifiers by clicking on them, see~\cite{Hetzl13Understanding}
for a detailed description.


\section{Cut-Elimination by Resolution (CERES)}


\subsection{First-Order Logic}

\begin{meta}
The ceres-functionality should be demonstrated by an example
session using extractStruct, structToClausesList, prover9 etc.
\end{meta}


\subsection{Higher-Order Logic}


\subsection{Schematic First-Order Logic}


\section{Cut-Introduction}

The cut-introduction algorithm as described in \cite{HetzlXXAlgorithmic} is
implemented in GAPT for introducing a single $\Pi_1$-cut into a sequent calculus
proof. In this section we show the commands, step by step, that need to be
executed for this algorithm. We will use as input one of the proofs generated by
the system, namely, \texttt{LinearExampleProof(9)}. But the
user can also, for example, write his own proofs in
\texttt{hlk}\footnote{\url{http://www.logic.at/hlk/}} and input these files to
the system. 

Make sure that the example proof sequences are available in the current CLI
session if you have not done so already by entering
\begin{lstlisting}
scala> :load ../examples/ProofSequences.scala
\end{lstlisting}
%
First of all, we instantiate the desired proof and store this in a variable:
\begin{lstlisting}
scala> val p = LinearExampleProof(9) 
\end{lstlisting}

You will see that a big string representing the proof is printed. If desired,
you can view this proof using \texttt{prooftool}. 
It is possible to see some information about a proof on the command line by calling:
\begin{lstlisting}
scala> printProofStats(p)
------------- Statistics ---------------
Cuts: 0
Number of quantifier rules: 9
Number of rules: 28
Quantifier complexity: 9
----------------------------------------
\end{lstlisting}
Now we need to
extract the terms used to instantiate the $\forall$ quantifiers of the
end-sequent:
\begin{lstlisting}
scala> val ts = extractTerms(p)
The term set contains 9 terms.
\end{lstlisting}
The system indicates how many terms were extracted, which is nine for this case,
as expected. The next step consists in computing grammars that generate this
term set:
\begin{lstlisting}
scala> val grms = computeGrammars(ts)
693 grammars found.
\end{lstlisting}
The number of grammars found is shown, 693 in this case. They are ordered
by size, and one can see the first ones by calling:
\begin{lstlisting}
scala> seeNFirstGrammars(grms, 5)
0. \{ tuple1(s(s(s(\(\alpha\))))), tuple1(\(\alpha\)), tuple1(s(s(s(s(s(s(\(\alpha\)))))))) \} o \{ 0, s(0), s(s(0)) \}
(size = 6)
1. \{ tuple1(s(\(\alpha\))), tuple1(s(s(\(\alpha\)))), tuple1(\(\alpha\)) \} o \{ 0, s(s(s(0))), s(s(s(s(s(s(0)))))) \}
(size = 6)
2. \{ tuple1(s(0)), tuple1(\(\alpha\)), tuple1(s(s(\(\alpha\)))), tuple1(s(s(s(\(\alpha\))))) \} o \{ 0, s(s(s(s(0)))), 
s(s(s(s(s(0))))) \}(size = 7)
3. \{ tuple1(s(s(0))), tuple1(\(\alpha\)), tuple1(s(s(s(\(\alpha\))))), tuple1(s(\(\alpha\))) \} o \{ 0, s(s(s(s(0)))), 
s(s(s(s(s(0))))) \}(size = 7)
4. \{ tuple1(\(\alpha\)), tuple1(s(s(\(\alpha\)))), tuple1(s(s(s(\(\alpha\))))), tuple1(s(\(\alpha\))) \} o \{ 0, s(s(s(s(0)))), 
s(s(s(s(s(0))))) \}(size = 7)

Note that the function symbols 'tuplei' are inserted by the system as part of the algorithm.
\end{lstlisting}
This will print on the screen the first 5 grammars, and we can choose which one
to use for compressing the proof, in this case we take the second one:
\begin{lstlisting}
scala> val g = grms(1)
\end{lstlisting}
Given the end-sequent of the proof and a grammar, the extended Herbrand sequent can be computed:
\begin{lstlisting}
scala> val ehs = generateExtendedHerbrandSequent(p.root, g)
ehs: at.logic.algorithms.cutIntroduction.ExtendedHerbrandSequent = 
P(0), (P(s(\(\alpha\)))\(\impl\)P(s(s(\(\alpha\))))), (P(s(s(\(\alpha\))))\(\impl\)P(s(s(s(\(\alpha\)))))),(P(\(\alpha\))\(\impl\)P(s(\(\alpha\)))),
(X(\(\alpha\))\(\impl\)(X(s(s(s(s(s(s(0)))))))\(\wedge\)(X(s(s(s(0))))\(\wedge\)X(0)))) :- 
P(s(s(s(s(s(s(s(s(s(0)))))))))),
\end{lstlisting}
As it was shown in \cite{HetzlXXAlgorithmic}, the cut-introduction problem has a
canonical solution:
\begin{lstlisting}
scala> val cs = computeCanonicalSolution(p.root, g)
Note that the clauses that do not contain the eigenvariable were already removed.
cs: at.logic.language.fol.FOLFormula = 
\(\forall\)x.((P(x)\(\impl\)P(s(x)))\(\wedge\)((P(s(s(x)))\(\impl\)P(s(s(s(x)))))\(\wedge\)(P(s(x))\(\impl\)P(s(s(x))))))
\end{lstlisting}
The extended Herbrand sequent generated previously has the canonical solution as
default, but this solution can be improved.. 
\begin{lstlisting}
scala> minimizeSolution(ehs)
Previous solution: \(\forall\)x.((P(x)\(\impl\)P(s(x)))\(\wedge\)((P(s(s(x)))\(\impl\)P(s(s(s(x)))))\(\wedge\)(P(s(x))\(\impl\)P(s(s(x))))))
Improved solution: \(\forall\)x.(P(s(s(s(x))))\(\vee \neg\)P(x))
\end{lstlisting}
Finally, the proof with cut is constructed:
\begin{lstlisting}
scala> val fp = buildProofWithCut(ehs)
\end{lstlisting}
In order to compare this with the initial proof, one can again count
the number of rules:
\begin{lstlisting}
scala> printProofStats(fp)
------------- Statistics ---------------
Cuts: 1
Number of quantifier rules: 7
Number of rules: 25
Quantifier complexity: 6
----------------------------------------
\end{lstlisting}
We showed how to run the cut-introduction algorithm step by step. There is,
though, a command comprising all these steps:
\begin{lstlisting}
scala> val fp2 = cutIntro(p)
\end{lstlisting}
Regarding the choice of the grammar, this command will compute the proofs with
all minimal grammars and choose the smallest one (with respect to the number of rules).


\section{Miscellaneous}

The method \texttt{printProofStats: LKProof $\rightarrow$ Unit}
takes a proof and prints some statistics about it, namely the number
of unary, binary and cut rules. If you have a variable storing the result of 
\texttt{loadProofs(...)}, one way to get the first proof of the list is: 
\texttt{proof.head.\_2}. If \texttt{proof} is the one defined on the previous 
example:

\begin{lstlisting}
scala> printPoofStats(proof.head._2)
unary: 2
binary: 1
cuts: 0
\end{lstlisting}

Of course you can always assing this to a variable and use it as a parameter:

\begin{lstlisting}
scala> var theproof = proof.head._2
theproof: at.logic.calculi.lk.base.LKProof = ExistsRightRuleType( :- $\exists$(x.($\forall$(y.(((a P x) $\wedge$ (a Q y)))))), ForallRightRuleType( :- $\forall$(y.(((a P b) $\wedge$ (a Q y)))), AndRightRuleType( :- ((a P b) $\wedge$ (a Q \beta)), InitialRuleType( :- (a P b)), InitialRuleType( :- (a Q \beta)))))

scala> printPoofStats(theproof)
unary: 2
binary: 1
cuts: 0
\end{lstlisting}


\vfill
\pagebreak
\begin{appendix}

\section{The Sequent Calculus}\label{app:sequent_calculus}

This section defines the rule systems used in GAPT. The rules can be constructed
via Scala-classes, which create the underlying data structure.

\subsection{First-order LK}

The rules of first-order LK are listed below. Proof trees are constructed top-down, starting with axioms and with each rule introducing new inferences. The constructing functions do perform sporadic checks, but in general, these do not guarantee well-formed proofs and the burden of correctness lies upon the programmer using them. Due to the top-down construction, quantification and equational rules are {\em especially} brittle and the programmer must himself take care to replace only the desired terms in a formula. Specifically, the programmer must supply the result of the replacement (the \texttt{main}-argument), which is accepted by the rule constructors without question.

\subsubsection*{Axioms}

\begin{prooftree}
\AxiomCm{}
\RightLabelm{(\mt{Identity Axiom})}
\UnaryInfCm{\Gamma, A \seq \Delta, A}
\end{prooftree}

\begin{prooftree}
\AxiomCm{}
\RightLabelm{(\mt{Reflexivity Axiom})}
\UnaryInfCm{\Gamma \seq \Delta, t = t}
\end{prooftree}

\subsubsection*{Cut}

  \begin{prooftree}
  \AxiomCm{\Gamma \seq \Delta, A}
  \AxiomCm{\Sigma, A \seq \Pi}
  \RightLabelm{(\mt{cut})}
  \BinaryInfCm{\Gamma, \Sigma \seq \Delta, \Pi}
  \end{prooftree}

\subsubsection*{Structural rules}

\begin{multicols}{2}

  \subsubsection*{Left rules}
  
  \begin{prooftree}
  \AxiomCm{\Gamma \seq \Delta}
  \RightLabelm{(\mt{w:l})}
  \UnaryInfCm{\Gamma, A \seq \Delta}
  \end{prooftree}
  
  \begin{prooftree}
  \AxiomCm{\Gamma, A, A \seq \Delta}
  \RightLabelm{(\mt{c:l})}
  \UnaryInfCm{\Gamma, A \seq \Delta}
  \end{prooftree}
  
  \subsubsection*{Right rules}
  
  \begin{prooftree}
  \AxiomCm{\Gamma \seq \Delta}
  \RightLabelm{(\mt{w:r})}
  \UnaryInfCm{\Gamma \seq \Delta, A}
  \end{prooftree}
  
  \begin{prooftree}
  \AxiomCm{\Gamma \seq \Delta, A, A}
  \RightLabelm{(\mt{c:r})}
  \UnaryInfCm{\Gamma \seq \Delta, A}
  \end{prooftree}

\end{multicols}

\newpage

\subsubsection*{Propositional rules}

\begin{multicols}{2}

  \subsubsection*{Left rules}

  \begin{prooftree}
  \AxiomCm{\Gamma, A \seq \Delta}
  \RightLabelm{(\land\mt{l}_1)}
  \UnaryInfCm{\Gamma, A \land B \seq \Delta}
  \end{prooftree}
  
  \begin{prooftree}
  \AxiomCm{\Gamma, B \seq \Delta}
  \RightLabelm{(\land\mt{l}_2)}
  \UnaryInfCm{\Gamma, A \land B \seq \Delta}
  \end{prooftree}
  
  \begin{prooftree}
  \AxiomCm{\Gamma, A \seq \Delta}
  \AxiomCm{\Sigma, B \seq \Pi}
  \RightLabelm{(\lor\mt{:l})}
  \BinaryInfCm{\Gamma, \Sigma, A \lor B \seq \Delta, \Pi}
  \end{prooftree}
  
  \begin{prooftree}
  \AxiomCm{\Gamma \seq \Delta, A}
  \RightLabelm{(\lnot\mt{:l})}
  \UnaryInfCm{\Gamma, \neg A \seq \Delta}
  \end{prooftree}
  
  \begin{prooftree}
  \AxiomCm{\Gamma \seq \Delta, A}
  \AxiomCm{\Sigma, B \seq \Pi}
  \RightLabelm{(\impl\mt{:l})}
  \BinaryInfCm{\Gamma, \Sigma, A \impl B \seq \Delta, \Pi}
  \end{prooftree}

  \subsubsection*{Right rules}

  \begin{prooftree}
  \AxiomCm{\Gamma \seq \Delta, A}
  \AxiomCm{\Sigma \seq \Pi, B}
  \RightLabelm{(\land\mt{:r})}
  \BinaryInfCm{\Gamma, \Sigma \seq \Delta, \Pi, A \land B}
  \end{prooftree}
  
  \begin{prooftree}
  \AxiomCm{\Gamma \seq \Delta, A}
  \RightLabelm{(\lor\mt{:r}_1)}
  \UnaryInfCm{\Gamma \seq \Delta, A \lor B}
  \end{prooftree}
  
  \begin{prooftree}
  \AxiomCm{\Gamma \seq \Delta, B}
  \RightLabelm{(\lor\mt{:r}_2)}
  \UnaryInfCm{\Gamma \seq \Delta, A \lor B}
  \end{prooftree}
  
  \begin{prooftree}
  \AxiomCm{\Gamma, A\seq \Delta}
  \RightLabelm{(\lnot\mt{:r})}
  \UnaryInfCm{\Gamma \seq \Delta, \neg A}
  \end{prooftree}
  
  \begin{prooftree}
  \AxiomCm{\Gamma, A \seq B, \Delta}
  \RightLabelm{(\impl\mt{:r})}
  \UnaryInfCm{\Gamma \seq A \impl B, \Delta}
  \end{prooftree}

\end{multicols}

\subsubsection*{Quantification roles}

\begin{multicols}{2}

  \subsubsection*{Left rules}

  \begin{prooftree}
  \AxiomCm{\Gamma, A[t/x] \seq \Delta}
  \RightLabelm{(\forall\mt{:l})}
  \UnaryInfCm{\Gamma, \forall x A \seq \Delta}
  \end{prooftree}
  
  \begin{prooftree}
  \AxiomCm{\Gamma, A[y/x] \seq \Delta}
  \RightLabelm{(\exists\mt{:l})}
  \UnaryInfCm{\Gamma, \exists x A \seq \Delta}
  \end{prooftree}
  
  \subsubsection*{Right rules}
  
  \begin{prooftree}
  \AxiomCm{\Gamma \seq A[y/x] \Delta}
  \RightLabelm{(\forall\mt{:r})}
  \UnaryInfCm{\Gamma \seq \forall x A, \Delta}
  \end{prooftree}
  
  \begin{prooftree}
  \AxiomCm{\Gamma \seq A[t/x] \Delta}
  \RightLabelm{(\exists\mt{:r})}
  \UnaryInfCm{\Gamma \seq \exists x A, \Delta}
  \end{prooftree}

\end{multicols}

The variable $y$ must not occur free in $\Gamma$, $\Delta$ or $A$. The term $t$ must avoid variable capture, i.e. it must not contain free occurrences of variables bound in $A$.

\subsubsection*{Equational rules}

\begin{multicols}{2}

  \subsubsection*{Left rules}

  \begin{prooftree}
  \AxiomCm{\Gamma \seq \Delta, s=t}
  \AxiomCm{\Sigma, A[T/s] \seq \Pi}
  \RightLabelm{(\mt{=:l}_1)}
  \BinaryInfCm{\Gamma, \Sigma, A[T/t] \seq \Delta, \Pi}
  \end{prooftree}
  
  \begin{prooftree}
  \AxiomCm{\Gamma \seq \Delta, s=t}
  \AxiomCm{\Sigma, A[T/t] \seq \Pi}
  \RightLabelm{(\mt{=:l}_2)}
  \BinaryInfCm{\Gamma, \Sigma, A[T/s] \seq \Delta, \Pi}
  \end{prooftree}
  
  \subsubsection*{Right rules}
  
  \begin{prooftree}
  \AxiomCm{\Gamma \seq \Delta, s=t}
  \AxiomCm{\Sigma \seq \Pi, A[T/s]}
  \RightLabelm{(\mt{=:r}_1)}
  \BinaryInfCm{\Gamma, \Sigma \seq \Delta, \Pi, A[T/t]}
  \end{prooftree}
  
  \begin{prooftree}
  \AxiomCm{\Gamma \seq \Delta, s=t}
  \AxiomCm{\Sigma, \seq \Pi, A[T/t]}
  \RightLabelm{(\mt{=:r}_2)}
  \BinaryInfCm{\Gamma, \Sigma \seq \Delta, \Pi, A[T/s]}
  \end{prooftree}

\end{multicols}


\begin{meta}
 TODO: what about LKsk? higher order LK?
\end{meta}

\section{The Resolution Calculus}

\begin{meta}
 TODO: give formal definition of our resolution calculus
\end{meta}

\section{Command Reference}

\begin{meta}
Shall we really keep a command reference in the user manual? This may make more sense
as \texttt{help <command>} in the CLI.
\end{meta}

\paragraph{\textbf{prover9: List[Sequent],Seq[Clause] $\rightarrow$ Option[ResolutionProof[Clause]]}}
sends the input clause set (given as either List[Sequent] or Seq[Clause]) to prover9. Returns
the resolution proof obtained from replaying the output proof of prover9,
see~\cite{Dunchev12System} for details.

\paragraph{\textbf{miniSATsolve: HOLFormula $\rightarrow$ Option[Interpretation]}}
Searches a model for a quantifier-free formula using the MiniSAT SAT Solver.
Returns None if unsatisfiable, and Some(Interpretation) otherwise.

\paragraph{\textbf{miniSATprove: HOLFormula $\rightarrow$ Boolean}}
Checks if a quantifier-free formula is valid using the MiniSAT SAT Solver.

\paragraph{\textbf{lkTolksk: LKProof $\rightarrow$ LKProof}}
This method takes a proof in classical logic (LK), such as one generated by HLK and 
loaded by the method \texttt{loadProofs}, and transforms it to a proof on the
calculus $\mathbf{LK}_{sk}$. This calculus was proposed to solve the problem of
Skolemization in higher-order logic, and it basically replaces eigenvariables
with Skolem terms. For more information, see \cite{Hetzl11CERES}.

\paragraph{\textbf{cutIntro: LKProof $\rightarrow$ LKProof}}
This method is the implementation of the cut-introduction algorithm described on
\cite{Hetzl2012}. It takes a cut-free proof in classical logic, automatically
computes a universally quantified cut formula and builds a new proof with this
cut.

\paragraph{\textbf{termsExtraction: LKProof $\rightarrow$ Map[FormulaOccurrence, List[List[FOLTerm]]]}}
A crucial part of the cut-introduction algorithm of \cite{Hetzl2012} is the
computation of the term set, which are the witnesses of the existential
quantifiers of the end-sequent of a proof. This method takes a proof and returns
a map. This map associates each existentially quantified formula of the end
sequent with a list of tuples of terms. These tuples will have the same size as
the number of quantifiers of the formula.

\paragraph{\textbf{regularize: LKProof $\rightarrow$ LKProof}}
{\color{red}TODO}

\paragraph{\textbf{createHOLExpression: String $\rightarrow$ HOLExpression
(Forall x1: (i -> (i -> i)) a(x1: (i -> (i -> i)), x2: i, c1: (i -> i)))}}
{\color{red}TODO}

\paragraph{\textbf{fsequent2sequent: FSequent $\rightarrow$ Sequent}}
{\color{red}TODO}

\paragraph{\textbf{deleteTautologies: List[FSequent] $\rightarrow$ List[FSequent]}}
{\color{red}TODO}

\paragraph{\textbf{removeDuplicates: List[FSequent] $\rightarrow$ List[FSequent]}}
{\color{red}TODO}

\paragraph{\textbf{unitResolve: List[FSequent] $\rightarrow$ List[FSequent]}}
{\color{red}TODO}

\paragraph{\textbf{removeSubsumed: List[FSequent] $\rightarrow$ List[FSequent]}}
{\color{red}TODO}

\paragraph{\textbf{normalizeClauses: List[FSequent] $\rightarrow$ List[FSequent]}}
{\color{red}TODO}

\paragraph{\textbf{writeLatex: List[FSequent], String $\rightarrow$ Unit}}
{\color{red}TODO}

\paragraph{\textbf{writeLabelledSequentListLatex: List[LabelledSequent], String $\rightarrow$ Unit}}
{\color{red}TODO}

\paragraph{\textbf{extractStruct: LKProof $\rightarrow$ Struct}}
Extracts a struct from a LKProof. A struct is referred to as a clause term in
\cite{Baaz2011}. I will give a quick definition of it. For more details and
for an explanation of this structure on the CERES method, please refer to
\cite{Baaz2011}, Chapter 6.

Given a proof with cuts, by removing all the inference rules that operate on end
sequent ancestors, we obtain a proof of the empty sequent (refutation). The
axioms of this refutation are represented by \textit{clause terms}. Clause
terms are $\{\oplus, \otimes\}$-terms over clause sets, and it's interpretation
is the following:

\begin{align*}
  |\mathcal{C}| &= \mathcal{C} \text{ if $\mathcal{C}$ is a set of clauses.}\\
   |X \oplus Y| &= |X| \cup |Y|\\
  |X \otimes Y| &= |X| \times |Y|
\end{align*}

where $\mathcal{C} \times \mathcal{D} = \{ C \circ D | C \in \mathcal{C} \wedge
D \in \mathcal{D}\}$ and if $S = \Gamma \vdash \Delta$ and $S' = \Pi \vdash
\Lambda$, $S \circ S' = \Gamma, \Pi \vdash \Delta, \Lambda.$

\paragraph{\textbf{structToClausesList: Struct $\rightarrow$ List[Sequent]}} computes
the standard characteristic clause set
from the struct, see~\cite[Section 4.2.1]{WoltzenlogelPaleo09General} for details.

\paragraph{\textbf{structToLabelledClausesList: Struct $\rightarrow$ List[LabelledSequent]}}
{\color{red}TODO}


% do we still need this section?
% \section{Running the system}
% 
% If you used svn to check out the source of GAPT from the website, after
% everything is downloaded you should see two folders: ``doc'' and ``source''. To
% compile the system, just go into the folder ``source'' and type:
% 
% \begin{lstlisting}[mathescape=false]
% $ mvn install
% \end{lstlisting}
% 
% Then you can go get something to eat, a coffee or anything more interesting. The
% whole system takes about 15 minutes (or more) to compile, and you don't want to keep
% staring at a screen printing non sense for that long. It's good to check,
% though, once in a while, if no errors occurred. If something wrong happened,
% there are somethings you can do (in that order):
% 
% \begin{enumerate}
% \item Make sure you have all required packages (Section \ref{sec:sysreq}).
% \item Try running \texttt{make clean} and then \texttt{make install} again.
% \item Send us an e-mail ({\color{red}TODO}: which e-mail?).
% \end{enumerate}
% 
% Probably the first thing you want to do after compiling your project is run it
% and make some tests. The easiest way to do this, is to run the scripts in the
% source folder. There are two bash scripts in this folder that you should care
% about. You may first want to make them executable:
% 
% \begin{lstlisting}[mathescape=false]
% $ chmod +x cli.sh
% $ chmod +x gui.sh
% \end{lstlisting}
% 
% What do each of these scripts do?
% 
% \texttt{cli.sh}: This will start GAPT inside a scala iteractive shell.
% 
% \texttt{gui.sh}: This will start ProofTool, an interface to visualize proofs.
% 
% If for any reason these scripts do not work, no need for crying.
% The jars for these applications are created during compilation
% and it's just a matter of finding the directory where they are to execute them.
% 
% The jar for GAPT is created inside the directory \texttt{cli/target} (you
% should use the one with dependencies) and you can execute it like this:
% 
% \begin{lstlisting}[mathescape=false]
% $ java -jar cli-XX-SNAPSHOT-jar-with-dependencies.jar
% \end{lstlisting}
% 
% The jar for ProofTool is created inside the directory
% \texttt{gui/prooftool/target} (also use the one with dependencies) and again you
% can execute it like this:
% 
% \begin{lstlisting}[mathescape=false]
% $ java -jar prooftool-XX-SNAPSHOT-jar-with-dependencies.jar
% \end{lstlisting}

\end{appendix}

\vfill
\pagebreak

\bibliographystyle{plain}
\bibliography{references}
\end{document}
