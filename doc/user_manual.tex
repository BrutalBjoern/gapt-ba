\documentclass[a4paper, 11pt]{report}
\usepackage[utf8]{inputenc}
\usepackage{listings}
\usepackage{color}
\usepackage{amsmath}
\usepackage{url}
\usepackage{alltt}

\newcommand{\seq}{\vdash}	% the sequent sign
\newcommand{\impl}{\supset}

% Using a report without chapters
\renewcommand*\thesection{\arabic{section}}

\lstset{
  basicstyle=\footnotesize\ttfamily,
  breaklines=true,
  frame=lines
}

\title{GAPT - Generic Architecture for Proof Transformations \\ \vspace{0.5cm} User's manual}
% Whenever you contribute for this manual, include your name here
% (alphabetically by last name) :)
\author{
Stefan Hetzl - \texttt{hetzl@logic.at}\\
Giselle Reis - \texttt{giselle@logic.at}
}

\begin{document}
\maketitle

\section{Introduction}

Talk about the system, the motivation, the ceres method and the necessary
previous knowledge.

Required knowledge:
\begin{itemize}
  \item Proof theory
  \item Cut-elimination
  \item CERES method
\end{itemize}

\section{System Requirements}
\label{sec:sysreq}

To run GAPT you need to have these installed:

\begin{itemize}
\item Java JDK - 1.6.0 or higher (we had problems with OpenJDK, but you can try
it at your own risk)
\item Maven2 - 2.2.1 or higher
\item Subversion (a.k.a. svn)
\item Prover9 (\url{http://www.cs.unm.edu/~mccune/mace4/download/}) - make sure
the commands \texttt{prover9}, \texttt{prooftrans} and \texttt{tptp\_to\_ladr}
are available.
\end{itemize}

\section{Running the system}

If you used svn to check out the source of GAPT from the website, after
everything is downloaded you should see two folders: ``doc'' and ``source''. To
compile the system, just go into the folder ``source'' and type:

\begin{lstlisting}
$ mvn install
\end{lstlisting}

Then you can go get something to eat, a coffee or anything more interesting. The
whole system takes about 15 minutes (or more) to compile, and you don't want to keep
staring at a screen printing non sense for that long. It's good to check,
though, once in a while, if no errors occurred. If something wrong happened,
there are somethings you can do (in that order):

\begin{enumerate}
\item Make sure you have all required packages (Section \ref{sec:sysreq}).
\item Try running \texttt{make clean} and then \texttt{make install} again.
\item Send us an e-mail ({\color{red}TODO}: which e-mail?).
\end{enumerate}

Probably the first thing you want to do after compiling your project is run it
and make some tests. The easiest way to do this, is to run the scripts in the
source folder. There are two bash scripts in this folder that you should care
about. You may first want to make them executable:

\begin{lstlisting}
$ chmod +x cli.sh
$ chmod +x gui.sh
\end{lstlisting}

What do each of these scripts do?

\texttt{cli.sh}: This will start GAPT inside a scala iteractive shell.

\texttt{gui.sh}: This will start ProofTool, an interface to visualize proofs.

If for any reason these scripts do not work, no need for crying.
The jars for these applications are created during compilation
and it's just a matter of finding the directory where they are to execute them.

The jar for GAPT is created inside the directory \texttt{cli/target} (you
should use the one with dependencies) and you can execute it like this:

\begin{lstlisting}
$ java -jar cli-XX-SNAPSHOT-jar-with-dependencies.jar
\end{lstlisting}

The jar for ProofTool is created inside the directory
\texttt{gui/prooftool/target} (also use the one with dependencies) and again you
can execute it like this:

\begin{lstlisting}
$ java -jar prooftool-XX-SNAPSHOT-jar-with-dependencies.jar
\end{lstlisting}

{\color{red}TODO}: describe how to download the jars from the site and run them. Easier than
installing the system...

\section{Some useful \texttt{scala} tips}

Since GAPT runs in an interactive \texttt{scala} shell, here are some useful things you
should know about it.

If you want to assign the result of a method to a variable, use: 
\texttt{var v = method(args)}. Otherwise, the system will create a variable by 
itself. You can see it's name and type before the description, like this:

With a variable name:

\begin{lstlisting}
scala> var i = 12
i: Int = 12
\end{lstlisting}

Without a variable name (in this case, the system created the variable \texttt{res12}):

\begin{lstlisting}
scala> 12
res12: Int = 12
\end{lstlisting}

To see the value of a variable, just type it’s name and press enter.

The elements of a list in Scala are indexed using parenthesis. So if \texttt{lst} 
is a list, the first element is obtained by \texttt{lst(0)}. It is also possible 
to use the methods \texttt{lst.head} (for the first element) and \texttt{lst.tail} 
(for the rest of the list).

The elements of a tuple \texttt{t} of size $n$ are accessed by the methods 
\texttt{t.\_1}, \texttt{t.\_2}, … , \texttt{t.\_n}.

To quit the interactive shell, just type \texttt{:quit} and press enter.

\section{Top level commands}

After executing the client (\texttt{cli.sh}), you will get to an interactive \texttt{scala} shell 
from where it is possile to run the operations GAPT provides. You can see all of
them by typing \texttt{ceresHelp} and pressing enter. Here they are (as of
28/08/2012):

%% TODO: update
\begin{lstlisting}
Available commands:

File Input/Output:
  loadProofDB: String => ProofDatabase - load proofdatabase from xml file
  loadProofs: String => List[(String, LKProof)] - load proofs from xml file as name/value pairs
  exportXML: List[Proof], List[String], String => Unit
  exportTPTP: List[Proof], List[String], String => Unit

Parsing:
  parse.fol: String => FOLFormula - example: "Forall x Imp P(x,f(x)) Exists y P(x,y)"
  parse.hol: String => HOLExpression
  parse.slk: String => Map[String, Pair[LKProof, LKProof]]

Automated Deduction:
  refuteFOL: Seq[Clause] => Option[ResolutionProof[Clause]] - call internal resolution prover TAP
  refuteFOLI: Seq[Clause] => Option[ResolutionProof[Clause]] - simple interactive refutation
  prover9: List[Sequent],Seq[Clause] => Option[ResolutionProof[Clause]] - call prover9
  proveProp: FSequent => Option[LKProof] - tableau-like proof search for propositional logic

Proof Theory:
  skolemize: LKProof => LKProof - skolemize the input proof

Cut-Elimination by Resolution:
  extractStruct: LKProof => Struct
  structToClausesList: Struct => List[Sequent]
  structToLabelledClausesList: Struct => List[LabelledSequent]

Cut-Introduction:
  cutIntro: LKProof => LKProof
  termsExtraction: LKProof => Map[FormulaOccurrence, List[List[FOLTerm]]] - extract the witnesses of the existential quantifiers of the end-sequent of a proof
  termsExtractionFlat: LKProof => Set[FOLTerm] - extract the witnesses of the existential quantifiers of the end-sequent of a proof (as a ,,flat'' set)

Proof Examples:
  LinearExampleTermset: Int => Set[FOLTerm] - construct the linear example termset for cut-introduction
  LinearExampleProof: Int => LKProof - construct the linear example proof for cut-introduction
  SquareDiagonalExampleProof: Int => LKProof - construct the square (diagonal) example proof for cut-introduction
  SquareEdgesExampleProof: Int => LKProof - construct the square (edges) example proof for cut-introduction
  SumExampleProof: Int => LKProof - construct the sum example proof for cut-introduction

Visualization:
  prooftool: LKProof => Unit - visualize proof in prooftool

Uncategorized:
  regularize: LKProof => LKProof
  printProofStats: LKProof => Unit
  lkTolksk: LKProof => LKProof
  createHOLExpression: String => HOLExpression (Forall x1: (i -> (i -> i)) a(x1: (i -> (i -> i)), x2: i, c1: (i -> i)))
  fsequent2sequent: FSequent => Sequent
  deleteTautologies: List[FSequent] => List[FSequent]
  removeDuplicates: List[FSequent] => List[FSequent]
  unitResolve: List[FSequent] => List[FSequent]
  removeSubsumed: List[FSequent] => List[FSequent]
  normalizeClauses: List[FSequent] => List[FSequent]
  writeLatex: List[FSequent], String => Unit
  writeLabelledSequentListLatex: List[LabelledSequent], String => Unit
\end{lstlisting}

In the following subsections we will (try to) describe what each of these are.

The syntax for calling the methods are always the name of the method followed by
the arguments in parenthesis, separated by commas if there's more than one.

\subsection{File Input/Output}

\paragraph{\textbf{loadProofs: String $\rightarrow$ List[(String, LKProof)]}}
This method will take as an argument a string that represents the path of a file
containing an xml proof (generated by HLK for example), and will return a list 
of pairs. It expects a file (the string of a proof will not work) and you can use 
the relative path. On the list returned, each pair is composed of a string and 
an object representing a proof within the system. The string is the name of the 
proof defined on the xml file, and it returns a list for the case that this file
contains more than one proof.

Example:
\begin{lstlisting}[mathescape]
scala> var proof = loadProofs("integration_tests/misc_test/src/test/resources/sk2.xml")
proof: List[(String, at.logic.calculi.lk.base.LKProof)] = List((p,ExistsRightRuleType( :- $\exists$(x.($\forall$(y.(((a P x) $\wedge$ (a Q y)))))), ForallRightRuleType( :- $\forall$(y.(((a P b) $\wedge$ (a Q y)))), AndRightRuleType( :- ((a P b) $\wedge$ (a Q \beta)), InitialRuleType( :- (a P b)), InitialRuleType( :- (a Q \beta)))))))
\end{lstlisting}

\paragraph{\textbf{loadProofDB: String $\rightarrow$ ProofDatabase}}
{\color{red}TODO}

\paragraph{\textbf{exportXML: List[Proof], List[String], String $\rightarrow$ Unit}}
Exports several proofs to an XML file. The first argument is the list of proofs,
the second is the list of the names of the proofs and the third is the name of
the file that will be written. The file path can also be specified and it's
relative to where the program is being executed.

The code below exports the proofs in the variables $p1$ and $p2$ to the file
\texttt{result.xml} with names ``First proof'' and ``Second proof'',
respectively.
\begin{lstlisting}
scala> exportXML(p1::p2::Nil, "First proof"::"Second proof"::Nil, "result.xml")
\end{lstlisting}

\paragraph{\textbf{exportTPTP: List[Proof], List[String], String $\rightarrow$ Unit}}
{\color{red}TODO}

\subsection{Parsing}

\paragraph{\textbf{parse.fol: String $\rightarrow$ FOLFormula}}
{\color{red}TODO}

\paragraph{\textbf{parse.hol: String $\rightarrow$ HOLFormula}}
{\color{red}TODO}

\paragraph{\textbf{parse.fol: String $\rightarrow$ Map[String, Pair[LKProof, LKProof]]}}
{\color{red}TODO}

\subsection{Automated Deduction}
  
\paragraph{\textbf{refuteFOL: Seq[Clause] $\rightarrow$ Option[ResolutionProof[Clause]]}}
{\color{red}TODO}

\paragraph{\textbf{refuteFOLI: Seq[Clause] $\rightarrow$ Option[ResolutionProof[Clause]]}}
{\color{red}TODO}

\paragraph{\textbf{prover9: List[Sequent],Seq[Clause] $\rightarrow$ Option[ResolutionProof[Clause]]}}
sends the input clause set (given as either List[Sequent] or Seq[Clause]) to prover9. Returns
the resolution proof obtained from replaying the output proof of prover9,
see~\cite{Dunchev12System} for details.

\paragraph{\textbf{proveProp: FSequent $\rightarrow$ Option[LKProof]}}
{\color{red}TODO}

\subsection{Proof Theory}

\paragraph{\textbf{skolemize: LKProof $\rightarrow$ LKProof}}
{\color{red}TODO}

\paragraph{\textbf{extractInterpolant: ( p: LKProof, npart: Set[FormulaOccurrence],
ppart: Set[FormulaOccurrence] ) $\rightarrow$ HOLFormula}}

Extracts an interpolant from a cut-free proof in the sequent calculus. The
implementation is based on Lemma 6.5 of~\cite{Takeuti87Proof}. The method expects
a proof p and an arbitrary partition of the end-sequent $\Gamma \seq \Delta$ of p into a 
``negative part'' $\Gamma_1\seq\Delta_1$ and a ``positive part'' $\Gamma_2 \seq \Delta_2$.
It returns a formula $I$ s.t.\ $\Gamma_1\seq\Delta_1, I$ and $I,\Gamma_2\seq\Delta_2$
are provable and $I$ contains only such predicate symbols that appear in both, $\Gamma_1\seq\Delta_1$
and $\Gamma_2\seq\Delta_2$. For example, suppose pr is a proof of $p \lor q \seq p, q$
by a single $\lor$-left inference, then you can compute an interpolant as follows:
\begin{lstlisting}
scala> val s = pr.root
s: at.logic.calculi.lk.base.Sequent = (p:o $\lor$ q:o) :- p:o, q:o

scala> val npart = Set( s.antecedent( 0 ), s.succedent( 0 ) )
npart: scala.collection.immutable.Set[at.logic.calculi.occurrences.FormulaOccurrence] = Set((p:o $\lor$ q:o)[10005], p:o[10006])

scala> val ppart = Set( s.succedent( 1 ) )
ppart: scala.collection.immutable.Set[at.logic.calculi.occurrences.FormulaOccurrence] = Set(q:o[10007])

scala> val I = extractInterpolant( pr, npart, ppart )
I: at.logic.language.hol.HOLFormula = ($\bot$:o $\lor$ q:o)
\end{lstlisting}

\paragraph{\textbf{extractHerbrandSequent: LKProof $\rightarrow$ Sequent}}

A proof in first order logic can be concisely represented with a Herbrand
sequent. Since this sequent is purely propositional, this is a way of reducing
first order logic to propositional logic. In short, if $\varphi$ is a proof and
$H$ is its Herbrand sequent, then $H$ contains all instances of the formulas in
the end-sequent of $\varphi$ and $H$ is a tautology. It is important to note
that $\varphi$ may not have strong quantifiers nor cuts on quantified formulas.
A more detailed explanation of Herbrand sequent extraction can be found in
\ref{Hetzl2008}. 

This method extracts a Herbrand sequent from a proof $\varphi$ that satisfies
the constraints. 
An execution and output of this command is shown below. The proof $p$ used is the
``Linear Example Proof'' for $n=4$\footnote{For the example proofs that are
automatically generated in the system, see Section \ref{sec:exampleproofs}.},
which is a proof without cuts of the end sequent:

$$P0, \forall x. (Px \rightarrow Psx) \vdash Ps^4 0$$

\begin{lstlisting}
scala> var hs = extractHerbrandSequent(p)
hs: at.logic.calculi.lk.base.types.FSequent =
([(P(s(s(s(0))))->P(s(s(s(s(0)))))), (P(s(s(0)))->P(s(s(s(0))))),
(P(s(0))->P(s(s(0)))), (P(0)->P(s(0))), P(0)], [P(s(s(s(s(0)))))])
\end{lstlisting}

Observe that this sequent (represented by two lists, one for the antecedent and
one for the succedent) is a tautology, is propositional and contains all
instances of $\forall x. (Px \rightarrow Psx)$ that must be used in a cut-free
proof.

\subsection{Cut-Elimination by Resolution}

% TODO: it would be nice to combine extractStruct,structToClausesList,prover9 etc.
%       to an example session for applying Ceres to a proof

\paragraph{\textbf{extractStruct: LKProof $\rightarrow$ Struct}}
Extracts a struct from a LKProof. A struct is referred to as a clause term in
\cite{Baaz2011}. I will give a quick definition of it. For more details and
for an explanation of this structure on the CERES method, please refer to
\cite{Baaz2011}, Chapter 6.

Given a proof with cuts, by removing all the inference rules that operate on end
sequent ancestors, we obtain a proof of the empty sequent (refutation). The
axioms of this refutation are represented by \textit{clause terms}. Clause
terms are $\{\oplus, \otimes\}$-terms over clause sets, and it's interpretation
is the following:

\begin{align*}
|\mathcal{C}| &= \mathcal{C} \text{ if $\mathcal{C}$ is a set of clauses.}\\
|X \oplus Y| &= |X| \cup |Y|\\
|X \otimes Y| &= |X| \times |Y|
\end{align*}

where $\mathcal{C} \times \mathcal{D} = \{ C \circ D | C \in \mathcal{C} \wedge
D \in \mathcal{D}\}$ and if $S = \Gamma \vdash \Delta$ and $S' = \Pi \vdash
\Lambda$, $S \circ S' = \Gamma, \Pi \vdash \Delta, \Lambda.$

\paragraph{\textbf{structToClausesList: Struct $\rightarrow$ List[Sequent]}} computes
the standard characteristic clause set
from the struct, see~\cite[Section 4.2.1]{WoltzenlogelPaleo09General} for details.

\paragraph{\textbf{structToLabelledClausesList: Struct $\rightarrow$ List[LabelledSequent]}}
{\color{red}TODO}

\subsection{Cut-Introduction}

\paragraph{\textbf{cutIntro: LKProof $\rightarrow$ LKProof}}
This method is the implementation of the cut-introduction algorithm described on
\cite{Hetzl2012}. It takes a cut-free proof in classical logic, automatically
computes a universally quantified cut formula and builds a new proof with this
cut. The example proofs listed in Section \ref{sec:exampleproofs} were
implemented to test this method and, in general, have a good compression rate.

\paragraph{\textbf{termsExtraction: LKProof $\rightarrow$ Map[FormulaOccurrence, List[List[FOLTerm]]]}}
A crucial part of the cut-introduction algorithm of \cite{Hetzl2012} is the
computation of the term set, which are the witnesses of the existential
quantifiers of the end-sequent of a proof. This method takes a proof and returns
a map. This map associates each existentially quantified formula of the end
sequent with a list of tuples of terms. These tuples will have the same size as
the number of quantifiers of the formula.

\paragraph{\textbf{termsExtractionFlat: LKProof $\rightarrow$ Set[FOLTerm]}}
{\color{red}TODO: I know what this does but what is it for exactly??} extract
the witnesses of the existential quantifiers of the end-sequent of a proof (as a
,,flat'' set)

\subsection{Proof Examples}
\label{sec:exampleproofs}

%% TODO: maybe we could put examples of these proofs for n=2
%% we can generate them and export to latex :)

\paragraph{\textbf{LinearExampleTermset: Int $\rightarrow$ Set[FOLTerm]}}
{\color{red}TODO}: construct the linear example termset for cut-introduction

\paragraph{\textbf{LinearExampleProof: Int $\rightarrow$ LKProof}}
{\color{red}TODO}: construct the linear example proof for cut-introduction

\paragraph{\textbf{SquareDiagonalExampleProof: Int $\rightarrow$ LKProof}}
{\color{red}TODO}: construct the square (diagonal) example proof for cut-introduction

\paragraph{\textbf{SquareEdgesExampleProof: Int $\rightarrow$ LKProof}}
{\color{red}TODO}: construct the square (edges) example proof for cut-introduction

\paragraph{\textbf{SumExampleProof: Int $\rightarrow$ LKProof}}
{\color{red}TODO}: construct the sum example proof for cut-introduction

\subsection{Visualization}

\paragraph{\textbf{prooftool: LKProof $\rightarrow$ Unit}}
{\color{red}TODO}: visualize proof in prooftool

\subsection{Uncategorized}

\paragraph{\textbf{printProofStats: LKProof $\rightarrow$ Unit}}
This method takes a proof and prints some statistics about it, namely the number
of unary, binary and cut rules. If you have a variable storing the result of 
\texttt{loadProofs(...)}, one way to get the first proof of the list is: 
\texttt{proof.head.\_2}. If \texttt{proof} is the one defined on the previous 
example:

\begin{lstlisting}
scala> printPoofStats(proof.head._2)
unary: 2
binary: 1
cuts: 0
\end{lstlisting}

Of course you can always assing this to a variable and use it as a parameter:

\begin{lstlisting}
scala> var theproof = proof.head._2
theproof: at.logic.calculi.lk.base.LKProof = ExistsRightRuleType( :- $\exists$(x.($\forall$(y.(((a P x) $\wedge$ (a Q y)))))), ForallRightRuleType( :- $\forall$(y.(((a P b) $\wedge$ (a Q y)))), AndRightRuleType( :- ((a P b) $\wedge$ (a Q \beta)), InitialRuleType( :- (a P b)), InitialRuleType( :- (a Q \beta)))))

scala> printPoofStats(theproof)
unary: 2
binary: 1
cuts: 0
\end{lstlisting}

\paragraph{\textbf{lkTolksk: LKProof $\rightarrow$ LKProof}}
This method takes a proof in classical logic (LK), such as one generated by HLK and 
loaded by the method \texttt{loadProofs}, and transforms it to a proof on the
calculus $\mathbf{LK}_{sk}$. This calculus was proposed to solve the problem of
Skolemization in higher-order logic, and it basically replaces eigenvariables
with Skolem terms. For more information, see \cite{Hetzl2011}.

\paragraph{\textbf{regularize: LKProof $\rightarrow$ LKProof}}
{\color{red}TODO}

\paragraph{\textbf{createHOLExpression: String $\rightarrow$ HOLExpression
(Forall x1: (i -> (i -> i)) a(x1: (i -> (i -> i)), x2: i, c1: (i -> i)))}}
{\color{red}TODO}

\paragraph{\textbf{fsequent2sequent: FSequent $\rightarrow$ Sequent}}
{\color{red}TODO}

\paragraph{\textbf{deleteTautologies: List[FSequent] $\rightarrow$ List[FSequent]}}
{\color{red}TODO}

\paragraph{\textbf{removeDuplicates: List[FSequent] $\rightarrow$ List[FSequent]}}
{\color{red}TODO}

\paragraph{\textbf{unitResolve: List[FSequent] $\rightarrow$ List[FSequent]}}
{\color{red}TODO}

\paragraph{\textbf{removeSubsumed: List[FSequent] $\rightarrow$ List[FSequent]}}
{\color{red}TODO}

\paragraph{\textbf{normalizeClauses: List[FSequent] $\rightarrow$ List[FSequent]}}
{\color{red}TODO}

\paragraph{\textbf{writeLatex: List[FSequent], String $\rightarrow$ Unit}}
{\color{red}TODO}

\paragraph{\textbf{writeLabelledSequentListLatex: List[LabelledSequent], String $\rightarrow$ Unit}}
{\color{red}TODO}

\section{ProofTool}
\label{sec:prooftool}

TO WRITE

\section{Algorithms}

In this section we describe how to run some algorithms implemented in GAPT, from
loading the proof until obtaining the final result.

\subsection{Cut-introduction}

The cut-introduction algorithm was described in \cite{cutIntro2013} and it is
implemented in gapt for introducing a single $\Pi_1$-cut into a sequent calculus
proof. In this section we show the commands, step by step, that need to be
executed for this algorithm. We will use as input one of the proofs generated by
the system, namely, \texttt{LinearExampleProof(9)}. But the
user can also, for example, write his own proofs in
\texttt{hlk}\footnote{\url{http://www.logic.at/hlk/}} and input these files to
the system. 

First of all, we instantiate the desired proof and store this in a variable:
{\small
\begin{alltt}
scala> val p = LinearExampleProof(9) 
\end{alltt}
}

You will see that a big string representing the proof is printed. If desired,
you can view this proof using \texttt{prooftool} \ref{sec:prooftool}. 
It is possible to see some information about a proof on the command line by calling:
{\small
\begin{alltt}
scala> printProofStats(p)
------------- Statistics ---------------
Cuts: 0
Number of quantifier rules: 9
Number of rules: 28
Quantifier complexity: 9
----------------------------------------
\end{alltt}
}
Now we need to
extract the terms used to instantiate the $\forall$ quantifiers of the
end-sequent:
{\small
\begin{alltt}
scala> val ts = extractTerms(p)
The term set contains 9 terms.
\end{alltt}
}
The system indicates how many terms were extracted, which is nine for this case,
as expected. The next step consists in computing grammars that generate this
term set:
{\small
\begin{alltt}
scala> val grms = computeGrammars(ts)
693 grammars found.
\end{alltt}
}
The number of grammars found is shown, 693 in this case. They are ordered
by size, and one can see the first ones by calling:
{\small
\begin{alltt}
scala> seeNFirstGrammars(grms, 5)
0. \{ tuple1(s(s(s(\(\alpha\))))), tuple1(\(\alpha\)), tuple1(s(s(s(s(s(s(\(\alpha\)))))))) \} o \{ 0, s(0), s(s(0)) \}
   (size = 6)
1. \{ tuple1(s(\(\alpha\))), tuple1(s(s(\(\alpha\)))), tuple1(\(\alpha\)) \} o \{ 0, s(s(s(0))), s(s(s(s(s(s(0)))))) \}
   (size = 6)
2. \{ tuple1(s(0)), tuple1(\(\alpha\)), tuple1(s(s(\(\alpha\)))), tuple1(s(s(s(\(\alpha\))))) \} o \{ 0, s(s(s(s(0)))), 
   s(s(s(s(s(0))))) \}(size = 7)
3. \{ tuple1(s(s(0))), tuple1(\(\alpha\)), tuple1(s(s(s(\(\alpha\))))), tuple1(s(\(\alpha\))) \} o \{ 0, s(s(s(s(0)))), 
   s(s(s(s(s(0))))) \}(size = 7)
4. \{ tuple1(\(\alpha\)), tuple1(s(s(\(\alpha\)))), tuple1(s(s(s(\(\alpha\))))), tuple1(s(\(\alpha\))) \} o \{ 0, s(s(s(s(0)))), 
   s(s(s(s(s(0))))) \}(size = 7)

Note that the function symbols 'tuplei' are inserted by the system as part of the algorithm.
\end{alltt}
}
This will print on the screen the first 5 grammars, and we can choose which one
to use for compressing the proof, in this case we take the second one:
{\small
\begin{alltt}
scala> val g = grms(1)
\end{alltt}
}
Given the end-sequent of the proof and a grammar, the extended Herbrand sequent can be computed:
{\small
\begin{alltt}
scala> val ehs = generateExtendedHerbrandSequent(p.root, g)
ehs: at.logic.algorithms.cutIntroduction.ExtendedHerbrandSequent = 
P(0), (P(s(\(\alpha\)))\(\impl\)P(s(s(\(\alpha\))))), (P(s(s(\(\alpha\))))\(\impl\)P(s(s(s(\(\alpha\)))))),(P(\(\alpha\))\(\impl\)P(s(\(\alpha\)))),
(X(\(\alpha\))\(\impl\)(X(s(s(s(s(s(s(0)))))))\(\wedge\)(X(s(s(s(0))))\(\wedge\)X(0)))) :- 
P(s(s(s(s(s(s(s(s(s(0)))))))))),
\end{alltt}
}
As it was shown in \cite{cutIntro2013}, the cut-introduction problem has a
canonical solution:
{\small
\begin{alltt}
scala> val cs = computeCanonicalSolution(p.root, g)
Note that the clauses that do not contain the eigenvariable were already removed.
cs: at.logic.language.fol.FOLFormula = 
\(\forall\)x.((P(x)\(\impl\)P(s(x)))\(\wedge\)((P(s(s(x)))\(\impl\)P(s(s(s(x)))))\(\wedge\)(P(s(x))\(\impl\)P(s(s(x))))))
\end{alltt}
}
The extended Herbrand sequent generated previously has the canonical solution as
default, but this solution can be improved, as demonstrated in Section
\ref{sec.improving}. 
{\small
\begin{alltt}
scala> minimizeSolution(ehs)
Previous solution: \(\forall\)x.((P(x)\(\impl\)P(s(x)))\(\wedge\)((P(s(s(x)))\(\impl\)P(s(s(s(x)))))\(\wedge\)(P(s(x))\(\impl\)P(s(s(x))))))
Improved solution: \(\forall\)x.(P(s(s(s(x))))\(\vee \neg\)P(x))
\end{alltt}
}
Finally, the proof with cut is constructed:
{\small
\begin{alltt}
scala> val fp = buildProofWithCut(ehs)
\end{alltt}
}
In order to compare this with the initial proof, one can again count
the number of rules:
{\small
\begin{alltt}
scala> printProofStats(fp)
------------- Statistics ---------------
Cuts: 1
Number of quantifier rules: 7
Number of rules: 25
Quantifier complexity: 6
----------------------------------------
\end{alltt}
%% :))))))
}
We showed how to run the cut-introduction algorithm step by step. There is,
though, a command comprising all these steps:
{\small
\begin{alltt}
scala> val fp2 = cutIntro(p)
\end{alltt}
}
Regarding the choice of the grammar, this command will compute the proofs with
all minimal grammars and choose the smallest one (with respect to the number of rules).

\bibliographystyle{plain}
\bibliography{references}
\end{document}
